\documentclass[a4paper,12pt]{amsbook}
\renewcommand
\baselinestretch{1.25}
% \usepackage{ccmap}
\usepackage{ctex}
\usepackage{mathrsfs}
\usepackage{amsmath, amsthm, amssymb, amscd}
\usepackage{latexsym}
\usepackage[numbers, sort&compress]{natbib}
\usepackage{amsmath}
\usepackage{amsfonts}
\usepackage{amssymb}
\usepackage{amsfonts}
% \usepackage{makeidx}
% \usepackage{graphicx}
%\usepackage{ccmap}
% \usepackage{tocbibind}
\usepackage{tikz}
\usepackage{dsfont}
\usepackage{hyperref}
\hypersetup{hypertex=true,
            colorlinks=true,
            linkcolor=blue,
            anchorcolor=red,
            citecolor=green,
            pdfauthor={裴幸智},%
            pdftitle={几个金融模型的参数估计},%
            pdfsubject={统计学},
            pdfkeywords={参数估计},
            pdfproducer={LaTeX},%
            pdfcreator={xeLaTeX}}
\allowdisplaybreaks
\newtheorem{theorem}{\indent 定理}[section]
\newtheorem{corollary}[theorem]{\indent 推论}
\newtheorem{proposition}[theorem]{\indent 命题}
\newtheorem{lemma}[theorem]{\indent 引理}
\newtheorem{nature}[theorem]{\indent 性质}
\newtheorem{definition}[theorem]{\indent 定义}
\newtheorem{remark}[theorem]{\indent 注记}
\newtheorem{example}[theorem]{\indent 例}
\newtheorem{Assumption}[theorem]{\indent 假定}
\newcommand{\headingstobeshown}{}
\renewcommand{\proofname}{\textbf{\indent 证明}}
% \renewcommand\subsectionformat{}
% \renewcommand\sectionformat{}
\def\Box{\hfill\square}
\def \tenrm{\rm}
\def\r{\mathbb{R}}
\def\z{\mathbb{Z}}
\def\n{\mathbb{N}}
\def\c{\hookrightarrow}
\def\cc{\hookrightarrow\hookrightarrow}
\def\m{\mbox}

%*******************************************************************************
\numberwithin{equation}{section}
\oddsidemargin=20pt \evensidemargin=0pt \textwidth=17.1cm
\textheight=22.5cm
\usepackage[top=2.84cm,bottom=2.54cm,left=3.47cm,right=2.9cm]{geometry}

%*******************************************************************************


\makeatletter \@addtoreset{equation}{section} \makeatother
\renewcommand\contentsname{目 \ 录}
\renewcommand\bibname{参考文献}
\begin{document}
\pagestyle{empty}
%\hoffset 1.0cm
\ \\
分\ 类\ 号 \underline{  \qquad\qquad\qquad   } \hspace{5cm}\ \ 密\  \  \ \ 级 \underline{  \ \ \ \ \ \ \ \ \ \ \ \ \ \ \  }\\
学校代码 \underline{ \  10414\qquad\  \  }\hspace{5.4cm}学\ \ \ 号 \underline{   201840100462   }\\
\vspace{2cm}
\begin{center}
{\heiti\kaishu\zihao{1}\ziju{0.8}江西师范大学}\vspace*{0.8cm}\\
{\heiti\songti\zihao{2}\ziju{0.6}硕士研究生学位论文}
\end{center}

\vspace*{1cm}

\begin{center}
\heiti\kaishu\zihao{2}\ \
{几个金融模型的参数估计}

\end{center}
\vspace*{4cm}
\ \\
\hspace*{3.0cm}{\kaishu\zihao{4}作\ 者\ 姓\ 名:\ \ \ \ \ \underline{\ \ \ \  \ \ \ \ 裴 \ 幸\ 智 \ \ \ \ \ \  \ }} \vspace*{0.8cm}\\
\hspace*{3.0cm}{\kaishu\zihao{4}指\ 导\ 教\ 师:\ \ \ \ \ \underline{\ \ \ \ \ \ \ \ 陈  \ \ 勇 \  \ \ \ \ \ \ \ \ \   }}\vspace*{0.8cm}\\
\hspace*{3.0cm}{\kaishu\zihao{4}专\ 业\ 名\ 称:\ \ \ \ \ \underline{\ \  \ \ \ \  统 \ 计  \ 学\ \ \ \ \ \ \  }}\vspace*{0.8cm}\\
\hspace*{3.0cm}{\kaishu\zihao{4}提\ 交\ 日\ 期:\ \ \ \ \ \underline{\ \ 二零二一 年 三 月\ \ }}\vspace*{0.8cm}\\
%%----------封面部分结束------------------------------%%

%%----------- 为了双面打印, 留一空白页
\newpage
\pagestyle{empty}

\newpage










%%-------------------申明部分开始------------------------------%%
\newpage
\begin{center}{\heiti\zihao{3}独\ 创\ 性\ 声\ 明 }\end{center}
\vspace*{0.5cm}

本人声明所呈交的学位论文是本人在陈勇导师指导下进行的研究工作及取得的研究成果. 据我所知,
除了文中特别加以标注和致谢的地方外,
论文中不包含其他人已经发表或撰写过的研究成果, 也不包含
为获得或其他教育机构的学位或证书而使用过的材料. 与我一同工作的同志对本研究所做的任何贡献
均已在论文中作了明确的说明并表示谢意.
\\

学位论文作者签名 :   \qquad\qquad\qquad \qquad\qquad\   签字日期 : \qquad 年 \qquad 月 \qquad 日

\vspace*{4cm}
\begin{center}{\heiti\zihao{3}学位论文版权使用授权书}\end{center}
\vspace*{0.5cm}


本学位论文作者完全了解江西师范大学研究生院有关保留、使用学位论文的规定, 有权保留并向国家有关部门或机构送交论文的复印件和磁盘,
允许论文被查阅和借阅. 本人授权江西师范大学研究生院可以将学位论文的全部或部分内容编入有关数据库进行检索, 可以采用影印、缩印或扫描等复制手段保存、汇编学位论文.

(保密的学位论文在解密后适用本授权书)
\\


学位论文作者签名 :   \qquad\qquad\qquad \qquad\qquad      \ \ 导师签名 :


签字日期 : \qquad \qquad 年 \qquad 月 \qquad 日\qquad\qquad\  签字日期 : \qquad 年 \qquad 月 \qquad 日




\newpage
\newpage\pagenumbering{roman}\thispagestyle{plain}\chapter*{摘 \ 要}
% \addcontentsline{toc}{chapter}{摘 \ \ \ \ 要}
%\begin{minipage}{12cm} %Given the width of this environment


{\zihao{4}\par
Ornstein-Uhlenbeck(简称为:O-U)过程、Vasicek模型是关于短期利率的数学模型,其作用是在随机波动的影响下来对金融市场作出判断。在实际应用中,模型的结构和参数选择是一项复杂且重要的工作。选取合理的参数对于描述短期利率的变动规律至关重要,其决定了模拟金融、通信等实际问题的准确性。已有的文献研究大多讨论由分数布朗运动、次分数布朗运动等特殊过程作为随机项驱动的随机微分方程。如:分数布朗运动驱动下的O-U过程、分数布朗运动驱动下的Vasicek模型、次分数布朗运动的O-U过程等。本文专注于在连续时间观测值[0,T]上的一般高斯过程驱动下的O-U过程和Vasciek模型,主要讨论在一般高斯过程驱动下的两个过程的参数估计问题。本文工作分为二个部分:第一部分研究在一般高斯过程驱动下的Vasicek 模型的漂移系数其两种形式下的估计量的强相合性和渐近正态性;第二部分研究在一般高斯过程驱动下的带周期均值的O-U 过程的参数估计量的强相合性和渐近正态性。



\bigskip

\bigskip

\bigskip
{\heiti\zihao{4} 关键词\ :}  {\songti\zihao{4} \  O-U过程;\ Vasciek模型;\ 高斯过程;\  参数估计; \ 渐近性质  }



%\end{minipage}



\newpage
%%%%%%%%%%%%%%%%%%%%%%%%%%%%%%%%%%%%%%%%%% End of 摘要 %%%%%%%%%%%%%%%
%%%%%%%%%%%%%%%%%%%%%%%%%%%
%%%%%%%%%%%%%%%%%%%%%%%%%%%%%%%%%%%%%%%%%% The Abstract of Thesis %%%
%%%%%%%%%%%%%%%%%%%%%%%%%%%%%%%%%%%%%%%
\thispagestyle{plain}\chapter*{Abstract}
% \addcontentsline{toc}{chapter}{\large Abstract}
The Ornstein-Uhlenbeck (referred to as O-U) process and the Vasicek model are mathematical models of short-term interest rates, whose role is to make judgments on financial markets under the influence of random fluctuations. In practical applications, the structure and parameter selection of the model is a complex and important task. The selection of reasonable parameters is very important to describe the changing law of short-term interest rates, which determines the accuracy of simulated finance, communications and other practical problems. Most of the existing literature studies discuss stochastic differential equations driven by special processes such as fractional Brownian motion and sub-fractional Brownian motion as random terms. Such as: O-U process driven by fractional Brownian motion, Vasicek model driven by fractional Brownian motion, O-U process of sub-fractional Brownian motion, etc. This paper focuses on the O-U process and Vasciek model driven by the general Gaussian process on continuous time observations [0,T], and mainly discusses the parameter estimation problem of the two processes driven by the general Gaussian process. The work of this paper is divided into two parts: the first part studies the strong consistency and asymptotic normality of the estimators of the Vasicek model driven by the general Gaussian process; the second part studies the general Gaussian process The strong consistency and asymptotic normality of the parameter estimators of the OU process with periodic mean driven.
\bigskip

\bigskip

\bigskip

 \textbf{\zihao{4}\textbf{Keywords}:} O-U process; \ Vasciek model; \ Gaussian process; \ parameter estimation; \ asymptotic properties

\newpage



%%%%%%%%%%%%%%%%%%%%%%%%%%%%%%%%%%%%%%%%%% Contents %%%%%%%%%%%%%%%
%%%%%%%%%%%%%%%%%%%%%%%%$
\tableofcontents
\newpage
%%%%%%%%%%%%%%%%%%%%%%%%%%%%%%%%%%%%%%%%%% End of Contents %%%%


%%%%%%%%%%%%%%%%%%%%%%%%%%%%%%%%%%%
\pagenumbering{arabic} \setcounter{page}{1}
\include{cschap1}



\chapter{引 \ 言}\renewcommand\headingstobeshown{ 引 \ 言}
\label{chapter:1}
一直以来,随机过程就被应用于社会学,物理,生物等模型的建立,现广泛应用于金融经济中。从理论上讲,随机过程的统计推断和模型应用同等重要。随着随机过程的广泛应用,随机微分方程也很广泛应用于金融模型的建立。所以在O-U 过程和Vasicek模型的应用中,其驱动过程的性质起着决定性的作用。由于随机微分方程的参数存在不确定性,参数的性质直接关系着方程的性质,那么研究参数估计的意义尤其重大。因此,这一方面的研究受到了广泛而密切的关注。

由\cite{Vasicek.O 1977} 提出的Vasicek模型在经济、金融、医学、生物和环境科学等许多领域有着广泛的应用。在经济领域,它被用来描述利率的波动,如\cite{Jingzhi.H 2012};在金融领域, 如{\cite{San. W 2020}中被用作随机投资模型。有关Vasicek 模型的统计推论在以往的文献中已经得到了很好的研究,例如,2008年 \cite{peng2008} 中用最小二乘估计和广义矩估计对Vasicek模型进行参数估计,且统计检验表明对数据的拟合效果较为理想。\cite{Fergusson K 2015}中提出了极大似然估计的方法估计漂移系数,且得出与市场评估一致的结论。

根据漂移系数的符号不同 $k>0$ 或 $k<0$,其研究的方法也会有所差异。分数布朗运动具有长相依性、自相似性和平稳增量性特殊性质,当将Vasicek 模型中的布朗运动换成分数布朗运动\cite{Xiao Yu 2017},且Hurst指数大于或等于二分之一时,分别证明了平稳情况 $k>0$, 爆炸情况 $k<0$, 以及零循环情况 $k=0$ 的渐近性质。在这三种情况下,考虑了最小二乘估计方法;并且当 $k>0$, 也考虑了 \cite{Hu.Y and Nualart 2010}中提出了的矩估计方法。

基于此, \cite{Xiao Yu 2017}将他们的工作扩展到由sub-fBm驱动下的Vasicek型模型。对于非遍历和零循环的情况,考虑了用最小二乘估计来估计参数。另外,其扩展到了更一般的自相似过程,例如Hermite过程。(请参见
\cite{Nourdin D(2006)})。

此外,当布朗运动被具有自相似性的高斯过程 \cite{Xiao Yu 2017}代替时,基于 $G$ 的某些条件,在非遍历 $k<0$ 情况下,研究了最小二乘估计量的相合性和渐近性质。2020年 \cite{Yong.C 2020}研究了更一般高斯过程驱动下的O-U过程,得出了最小二乘估计量和矩估计量的渐近正态性和 $Berry-Esseen$ 界的结论。

基于这一点,在第三部分中,我们考虑由一般高斯过程驱动的Vasicek  模型,该过程不具有自相似或具有平稳的增量,研究了当持久性参数k为正时的漂移系数的参数估计问题。

顺着这一研究路线,在本文第四章我们将针对在一般高斯过程驱动下的带有周期均值的O-U模型给出参数的相合性和渐近正态性。由于,在许多现实生活过程中(例如,在商品和能源定价过程,请参见\cite{Geman2005}。由于季节性模式或过程的长期趋势,恒定平均水平的假设是不够的。因此从纯理论的角度研究在一般高斯过程驱动下带周期均值的的O-U过程是非常必要的。


\chapter{预备知识}
\label{chapter:2}
%\setcounter{page}{1} \pagenumbering{arabic}  %修改2011.11.01
在这部分中,我们描述了有关高斯过程的随机积分推算的一些基本知识,并回顾了 \cite{Nourdin D(2006)}中有关多元积分的中心极限定理的主要结果,详情请参考 \cite{Yong.C 2020}。
\section{一般高斯过程}
\label{section:2.1}
在完整的概率空间 $(\Omega,\mathcal{F},P)$ 上定义,  $\mathcal{F}$ 由高斯族 $G$ 生成. 将 $G={G_{t},t\in[0,T]}$ 表示为连续中心化的高斯过程,并假设协方差函数 $R$ 是连续的。
\begin{equation}
\mathbb{E}(G_{t}G_{s})=R(s,t),s,t\in[0,T],
\end{equation}
令 $\varepsilon$ 表示 $[0,T]$ 上所有实值阶跃函数的空间, 希伯尔空间 $\mathfrak{H}$ 定义为赋予内积的 $\varepsilon$ 的闭包。
\begin{equation}
\langle \mathds{1}_{[a,b)},\mathds{1}_{[c,d)}\rangle_{\mathfrak{H}}=\mathbb{E}((G_{b}-G_{a})(G_{d}-G_{c}))
\end{equation}
如果 $G={G_{h}, h\in{\mathfrak{H}}}$ 作为概率空间上的等高斯过程, 则由希伯尔空间 $\mathfrak{H}$ 的元素索引, $G$ 为如下图所示的高斯随机变量族。
\begin{equation}
\mathbb{E}(G_{g}G_{h})=\left \langle g,h \right \rangle_{\mathfrak{H}},\forall g,h \in\mathfrak{H}
\end{equation}
下列命题是 \cite{Jolis M 2007} 中定理2.3的拓展, 该定理给出了希伯尔空间 $\mathfrak{H}$ 及其中引用的内积的表示。
\begin{proposition}
将 $\mathcal{V}_{[0,T]}$ 表示为 $[0,T]$ 上的有界变化函数集, 然后将 $\mathcal{V}_{[0,T]}$ 设为在 $\mathcal{V}$ 上的稠密函数,可以得到
\begin{equation}
  \langle f,g\rangle_\mathfrak{H}=\int_{[0,T]^{2}}R(t,s){v}_{f}(dt){v}_{g}(ds),      \forall f,g\in \mathcal{V}_{[0,T]},
\end{equation}
其中 ${v}_{g}$ 是与 $g^{0}$ 关联的Lebesgue-Stieljes测度,定义为
\begin{equation}
g^{0}= \left\{
\begin{array}{rcl}
g(x),&& {if \quad x\in [0,T]}\\
0,&& {otherwise}\\
\end{array}
\right.
\end{equation}
此外,如果协方差函数 $R(t,s)$ 满足假定1.1, 则
\begin{equation}
\langle f,g\rangle_\mathfrak{H}=\int_{[0,T]^{2}}R(t,s)\frac{\partial ^{2}R(t,s)}{\partial
t\partial s}dtds,                  \forall f,g\in \mathcal{V}_{[0,T]}
\end{equation}
\end{proposition}
 \begin{corollary}\label{tui2.1.2}
如果满足假定1.1, 则存在一个独立于 $T$ 的常数 $C>0$,这样对于所有 $s,t\geq 0$,
 \begin{equation}
 \mathbb{E}(G_{t}-G_{s})^{2}\leq C_{\beta}|t-s|^{2\beta}
 \end{equation}
当 $s=0$, 我们由 $E(G_{t}^{2})\leq C_{\beta}^{'}t^{2\beta}$。
 \begin{proof}
 \begin{equation}
 \begin{aligned}
 \mathbb{E}(G_{t}-G_{s})^{2}&=\int_{[s,t]^{2}}\frac{\partial ^{2}R(u,v)}{\partial u\partial v}dudv\\
 &\leq \int_{[s,t]^{2}}|u-v|^{2\beta-2}dudv+\int_{[s,t]^{2}}|uv|^{\beta-1}dudv
 &\leq C_{\beta}|t-s|^{2\beta}
 \end{aligned}
 \end{equation}
 因此,得出了需证明的结果。
 \end{proof}
 \end{corollary}
\begin{remark}
将$\mathfrak{H}^{\bigotimes p}$ 和 $\mathfrak{H}^{\bigodot p}$ 分别表示为希伯尔空间 $\mathfrak{H}$ 上的p维上的张量积和对称张量积. 令 $\mathcal H_p$ 是关于G的Wiener混沌。它定义为随机变量 ${H_{p}(G(h))h\in{\mathfrak{H}}}$产生的封闭的线性子空间 $L^{2}(\Omega)$, 其中 $H_{p}$ 是定义的p维Hermite多项式
\begin{equation}
H_{p}(x)=\frac{(-1)^{p}}{p!}e^{\frac{x^{2}}{2}}\frac{d^{p}}{dx^{p}}e^{-%
\frac{x^{2}}{2}},p\geq 1
\end{equation}
其中 $H_{0}(x)=1$。对于 $h\in H$, 我们有 $I_{p}(h^{\bigotimes p})=H_{p}(G(h))$ ,其中 $I_{p}(\cdot )$ 为广义的 Wiener-It$\widehat{o}$ 随机积分。然后映射 $I_{p}$ 在 $\mathfrak{H}^{\bigodot p}$ 和$H_{p}$之间提供了线性等距线。根据规定有 $H_{o}=R$ 和 $I_{0}(x)=x$ 。\par
我们设 $e_{k}$ 是希伯尔空间$\mathfrak{H}$ 中完备的标准正交系. 给定$ f\in \mathfrak{H}^{\bigodot m} $ 和 $ g\in \mathfrak{H}^{\bigodot n}$ ,$f$ 和 $g$ 之间的q维收缩在 $\mathfrak{H}^{m+n-2q}$ 上的一个元素可被定义为
\begin{equation}
f\bigotimes_{q}g=\overset{\infty }{\underset{i_{1},\cdot \cdot \cdot ,i_{q}=1}{\sum
}}\langle f,e_{i_{1}\bigotimes \cdot \cdot \cdot \bigotimes}e_{i_{q}}\rangle_{\mathfrak{H}^{\bigotimes{q}}}\bigotimes \langle g,e_{i_{1}\bigotimes \cdot \cdot\cdot \bigotimes}e_{i_{q}}\rangle_{\mathfrak{H}^{\bigotimes{q}}}, q=1,...,m\wedge n.
\end{equation}
然后,对于多重积分我们有如下乘积公式
\begin{equation}
I_{p}(g)I_{q}(h)=\underset{r=0}{\overset{p\wedge q}{\sum }}%
r!\tbinom{p}{r}\tbinom{q}{r}I_{p+q-2r}(g\widetilde{\bigotimes}_{r}h)
\end{equation}
\end{remark}
下列定理2.3被称为四阶矩定理,其为持久参数 $k$ 的渐近理论提供了充要条件。 (请参考\cite{Nualart D(2005)})
\begin{theorem}
令$n\geq 2$ 为固定整数,考虑元素 ${f_{T}, T>0}$ 的集合, 这样对于任意$T>0$,有$f_T\in \mathfrak{H}\bigodot n$,进一步假设
\begin{equation}
\underset{T\rightarrow \infty }{\lim }\mathbb{E}[I_{n}(f_{T})^{2}]=\underset{%
T\rightarrow \infty }{\lim }n!||f_{T}||_{\mathfrak{H}^{\bigotimes n}}^{2}=\sigma ^{2}
\end{equation}
那么下列条件是等价的:\par
(1)$\underset{T\rightarrow \infty }{\lim }\mathbb{E}[I_{n}(f_{T})^{4}]=3\sigma ^{4}$.\par
(2)对于每个 q=1,...,n-1,$\underset{T\rightarrow \infty }{\lim }\mathbb{E}||f_{T}\bigotimes f_{T}||_{\mathfrak{H}\bigotimes 2(n-q)}=0$.\par
(3)随着 $T$ 趋于无穷大, 第n个多重积分 $\{I_{n}(f_{T}),T\geq 0\}$ 收敛到高斯分布 $N(0,\sigma^{2})$.\par
\end{theorem}

接下来, 我们给出本文中将要用到的带周期均值的一些概念.

\section{带周期均值的一些概念}
\label{section:2.2}

现在我们回顾第四章将会运用到的基本知识。
我们引入利率 $r=(r(t))_{t\geq 0}$ 的一个有关扩散模型的流行模型,即Vasicek模型,其中 $W=(W_{t})_{t\geq 0}$ 是在给定某个随机基底上的标准维纳过程(布朗运动)
\begin{equation}\label{xixi}
  dr(t)=(\alpha-\beta r(t))dt+\gamma dW_{t}
\end{equation}
如果认为利率在某个常值水平 $\alpha/\beta$ 附近波动,那么Vasicek模型是完全自然的,当$r(t)<\alpha/\beta$,过程会呈现正漂移,而当 $r(t)>\alpha/\beta$时,漂移系数是负的;如果 $\alpha=0$,那么方程\ref{xixi}就会变成第四章将要研究的O-U过程。
经过许多债券利率性态的经验研究,不能认为存在某个固定值$(\alpha/\beta)$,因此,使得利率有回归这个值的倾向的现象称为周期均值。

\chapter{在一般高斯过程驱动下的Vasicek模型}
\label{chapter:3}
在金融市场中,Vasicek模型是一种描述利率演变的数学模型,其描述了只有在一种市场风险的情况下的利率变动过程。
Vasicek模型表示瞬时利率是遵循以下随机微分方程
\begin{align}\label{model}
dX_{t}=k(\mu -X_{t})dt+\sigma dG_{t},t\in \lbrack 0,T],T\geq 0,x_{0}=0,
\end{align}
其中$(G_{t})_{t\geq 0}$ 是一般的一维的高斯过程。波动率参数 $\sigma>0$ 可以通过幂变化法来估计。不失一般性, 这里假定 $\sigma=1$。假定只有一条轨迹 $(X_{t},t\geq 0)$, 我们构造最小二乘估计量和矩估计量,并研究其相合性和渐近正态性。\par
本章引用 \cite{Yong.C 2020},并对一般高斯过程协方差函数的二阶偏导形式做以下假设。
\begin{Assumption}
对于$\beta \in (\frac{1}{2},1)$, 任意 $t\neq s\in \lbrack 0,\infty )$ 的协方差函数为 $R(t,s)=E[G_{t}G_{s}]$ 。
\begin{align}
    \frac{\partial ^{2}}{\partial t\partial s}R(t,s)&=C_{\beta }|t-s|^{2\beta-2}+\Psi (t,s)\end{align}
以及  \begin{align}
    |\Psi (t,s)|&\leq C_{\beta }^{^{\prime }}|ts|^{\beta -1}
\end{align}
其中常数 $\beta ,C_{\beta }>0,C_{\beta }^{^{\prime }}\geq 0$ 不依赖于T。此外,对于任意 $t\geq 0$, 有$R(0,t)=0$。
\end{Assumption}
 我们可以看到分数布朗运动和其他一些高斯过程满足\ref{assumption:4.3.1}。 根据此假设,我们得到如下结果。
当 $k>0$ 时, $\mu$ 的估计量为连续时间样本均值 (see \cite{Hu.Y and Nualart 2010})。
\begin{equation}\label{4}
\widehat{\mu }=\frac{1}{T}\int_{0}^{T}X_{t}dt
\end{equation}
此外,参考 \cite{Xiao Yu 2017}, 当 $k>0$,
二阶矩估计量为
\begin{equation}
\widehat{k}=\left[ \frac{\frac{1}{T}\int_{0}^{T}X_{t}^{2}dt-(\frac{1}{T}%
\int_{0}^{T}X_{t}dt)^{2}}{C_{\beta }\Gamma (2\beta -1)}\right] ^{-\frac{1}{%
2\beta }}
\end{equation}
 $k$ 和 $\mu$ 的最小二乘估计量来自于下列函数的最小化
\begin{equation}
L(k,\mu )=\int_{0}^{T}(\overset{\cdot }{X_{t}}-k(\mu -X_{t}))^{2}dt
\end{equation}
求解方程式,我们可以得到 $k$ 和 $\mu$ 的最小二乘估计, 分别由 $\widehat{k}_{LS}$ 和 $\widehat{\mu }_{LS}$ 表示
\begin{equation}\label{109}
\widehat{k}_{LS}=\frac{X_{T}\int_{0}^{T}X_{t}dt-T\int_{0}^{T}X_{t}dX_{t}}{%
T\int_{0}^{T}X_{t}^{2}dt-(\int_{0}^{T}X_{t}dt)^{2}}
\end{equation}
\begin{equation}
\widehat{\mu }_{LS}=\frac{X_{T}\int_{0}^{T}X_{t}^{2}dt-\int_{0}^{T}X_{t}dX_{t}%
\int_{0}^{T}X_{t}dt}{X_{T}\int_{0}^{T}X_{t}dt-T\int_{0}^{T}X_{t}dX_{t}}
\end{equation}
其中 $\int_{0}^{T}X_{t}dX_{t}$ 可以理解为 It$\hat{o}$-Skorohod 积分 (\cite{Xiao Yu 2017})。\par
在本章节中, 我们将证明这四个估计量的强相合性和渐近性质,这些结果将在以下定理中说明。
\section{主要结果}
\label{section:3.1}
\begin{theorem}\label{theorem1}
当满足\ref{assumption:4.3.1}时, $\mu$ 和 $k$ 的最小二乘估计和矩估计是强相合的,即
\begin{equation}
\underset{T\rightarrow \infty }{\lim }\widehat{\mu }=\mu ,      \underset{%
T\rightarrow \infty }{\lim }\widehat{\mu }_{LS}=\mu ,      a.s.
\end{equation}
\begin{equation}
\underset{T\rightarrow \infty }{\lim }\widehat{k}=k,        \underset{T\rightarrow
\infty }{\lim }\widehat{k}_{LS}=k,     a.s.
\end{equation}
\end{theorem}
\begin{theorem}\label{theorem2}
假定\ref{assumption:4.3.1}是满足的,当$\beta\in(1/2,1)$时,且 $T\rightarrow \infty$, $T^{1-\beta }(\widehat{\mu }-\mu )$ 和 $T^{1-\beta }(\widehat{\mu }_{LS}-\mu )$ 是渐近正态的,即
\begin{equation}
T^{1-\beta }(\widehat{\mu }-\mu )\overset{law}{\rightarrow }N(0,\frac{1}{k^{2}})\\
T^{1-\beta }(\widehat{\mu }_{LS}-\mu )\overset{law}{\rightarrow }N(0,\frac{1}{k^{2}})\\
\end{equation}
当 $\beta \in (\frac{1}{2},\frac{3}{4})$, 且 $T\rightarrow \infty$ 时,$\sqrt{T}(\widehat{k}-k)$ 和 $\sqrt{T}(\widehat{k}_{LS}-k)$ 是渐近正态的, 即,
\begin{equation}
\sqrt{T}(\widehat{k}_{LS}-k)\overset{law}{\rightarrow }N(0,4ka^{2}\sigma _{\beta }^{2})\\
\sqrt{T}(\widehat{k}-k)\overset{law}{\rightarrow }N(0,\sigma _{\beta}^{2}k/4\beta ^{2})
\end{equation}
其中 $a=C_{\beta }\Gamma (2\beta -1)k^{-2\beta }$, $\sigma _{\beta }^{2}=(4\beta -1)[1+\frac{\Gamma (3-4\beta )\Gamma (4\beta-1)}{\Gamma (2\beta )\Gamma (2-2\beta )}]$。
\end{theorem}
\section{估计量的相合性}
\label{section:3.2}
\subsection{矩估计量的相合性}
\label{subsection:3.2.1}
若 $k>0$, 可以考虑 $k$ 和 $\mu$ 的估计量。在 \cite{Hu.Y and Nualart 2010} 中,在 $fBm_OU$ 模型中根据过程的平稳性和遍历特性为 $k$ 构造了一个新的估计量,在这里可以采用相同的方法。那么,我们首先考虑 $\hat\mu$ 的强相合性,(1.1)中模型的解由下式给出
\begin{equation}\label{23}
X_{t}=\mu (1-e^{-kt})+\int_{0}^{T}e^{-k(t-s)}dG_{s}
\end{equation}
因此,估计量 $\mu$ 是连续时间样本均值
\begin{equation}\label{24}
\widehat{\mu }=\frac{1}{T}\int_{0}^{T}X_{t}dt
\end{equation}
结合 (\ref{23}) 和 (\ref{24}), 这里可以将 $\widehat{\mu}$ 重新表示为,
\begin{equation}\label{25}
\widehat{\mu }=\frac{1}{T}\int_{0}^{T}(1-e^{-kt})\mu dt+\frac{1}{T}\int_{0}^{T}(\int_{0}^{t}e^{-k(t-s)}dG_{t})dt
\end{equation}
对于(\ref{25})中的第二项, 我们首先定义一些将在证明中使用的重要函数。
记作
$F_{T}=\int_{0}^{T}e^{-kt}\int_{0}^{t}e^{ks}dG_{s}dt$, 使用随机Fubini定理可得
\begin{equation}\label{34}
F_{T}=\int_{0\leq s\leq t\leq T}e^{-k(t-s)}dG_{s}dt=\int_{0}^{T}\frac{1}{k}(1-e^{-k(T-s )})dG_{s}
=G_{T}-Z_{T}
\end{equation}
其中 $Z_{T}=\int_{0}^{T}e^{-k(T-s)}dG_{s}$ 是Wiener-It$\hat{o}$ ,在本文的其余部分中,C将是一个独立于T的其值可能会有所差异的正常数。\par
\begin{remark}(see \cite{Yong.C 2020})
对于函数 $\phi(r)\in\mathcal{V}_{[0,T]}$, 我们定义两个准则为
\begin{equation}
\begin{aligned}
\left\Vert\phi\right\Vert_{\mathfrak{H}_{1}}^{2}=C_{\beta}\int_{[0,T]^{2}}\phi(r_{1}\phi(r_{2})|r_{1}-r_{2}|^{2\beta-2}dr_{1}dr_{2},\\
\left\Vert\phi\right\Vert_{\mathfrak{H}_{2}}^{2}=C_{\beta}^{'}\int_{[0,T]^{2}}|\phi(r_{1}\phi(r_{2})|(r_{1}-r_{2})^{\beta-1}dr_{1}dr_{2}\\
\end{aligned}
\end{equation}
对于在 $[0,T]^2$ 空间上的函数 $\varphi(r,s)$, 定义一个从 $\mathcal{V}_{[0,T]}^{\otimes {2}}$ 到 $\mathcal{V}_{[0,T]}$ 的算子,如下所示:
\begin{equation}
(K\varphi)(r)=\int_{0}^{T}|\varphi(r,u)|u^{\beta-1}du.
\end{equation}
\end{remark}
\begin{proposition}(see \cite{Yong.C 2020})
假设假定1.1成立, 那么对于任何 $\phi(r)\in\mathcal{V}_{[0,T]}$,
\begin{equation}\label{29}
|\left\Vert\phi\right\Vert_{\mathfrak{H}}^{2}-\left\Vert\phi\right\Vert_{\mathfrak{H}_{1}}^{2}|\leq\left\Vert\phi\right\Vert_{\mathfrak{H}_{2}}^{2},
\end{equation}
和 $\varphi,\psi \in (\mathfrak(V)_{0,T})^{\bigodot 2}$,有
\begin{equation}
\begin{aligned}\label{equation328}
|\left\Vert\phi\right\Vert_{\mathfrak{H}^{\bigotimes{2}}}^{2}-\left\Vert\phi\right\Vert_{\mathfrak{H}_{1}^{\bigotimes{2}}}^{2}|&\leq\left\Vert\phi\right\Vert_{\mathfrak{H}_{2}^{\bigotimes{2}}}^{2}+2C_{\beta}^{'}\left\Vert K\varphi\right\Vert_{\mathfrak{H}_{1}}^{2},\\
|\langle\varphi,\psi\rangle_{\mathfrak{H}^{\bigotimes{2}}}-\langle\varphi,\psi\rangle_{\mathfrak{H}_{1}^{\bigotimes{2}}}|&\leq|\langle\varphi,\psi\rangle_{\mathfrak{H}_{2}^{\bigotimes{2}}}|+2C_{\beta}^{'}|\langle K\varphi,K\psi\rangle_{\mathfrak{H}_{1}}|\\
\end{aligned}
\end{equation}
\end{proposition}
接下来的两个命题是当$0\leq t,s\leq T$ 时,分别关于$F_ {T} $的二阶矩和增量$ F_ {t} -F_ {s} $的二阶矩的渐近性质。 首先,我们需要引入如下引理。
\begin{lemma} \label{lemma1}
假设 $\beta \in (0,1)$, 这里存在一个常量 $C>0$ ,那么对于任意 $T\in \lbrack 0,\infty )$,
\begin{equation}
e^{-kT}\int_{0}^{T}e^{kr}r^{\beta -1}dr\leq C(1\wedge T^{\beta -1}).
\end{equation}
(见 \cite{Yong.C 2020}中lemma3.3)

\end{lemma}
\begin{proposition}
当$\beta\in (\frac{1}{2},1)$ 时, 可以得到
\begin{equation}\label{33}
E(F_{T}^{2})\leq C_{\beta }T^{2\beta }
\end{equation}
\begin{proof}
由于 It$\hat{o}$ 的等距性, 我们可以得到
\begin{equation}
\mathbb{E}[|F_{T}|^2]=\left\Vert f_{T}\right\Vert_{\mathfrak{H}}^2
\end{equation}
由不等式(\ref{29})可以推出
\begin{equation}
|\left\Vert f_{T}\right\Vert_{\mathfrak{H}}^{2}-\left\Vert f_{T}\right\Vert_{\mathfrak{H}_{1}}^{2}|\leq \left\Vert f_{T}\right\Vert_{\mathfrak{H}_{2}}^{2},
\end{equation}
那么,可得
\begin{equation}
0\leq \int_{0}^{T}(1-e^{-k(T-u)})u^{\beta-1}du \leq \int_{0}^{T}u^{\beta-1}du\leq CT^{\beta}
\end{equation}
因此有,
\begin{equation}
\left\Vert f_{T}\right\Vert_{\mathfrak{H}_{2}}^{2}=(\frac{1}{k}\int_{0}^{T}(1-e^{-k(T-u)})u^{\beta-1}du)^{2}\leq CT^{2\beta}
\end{equation}
同时,可以得到
\begin{equation}
\begin{aligned}
\left\Vert f_{T}\right\Vert_{\mathfrak{H}_{1}}^{2}&=\frac{1}{k^{2}}\int_{[0,T]^{2}}(1-e^{-k(T-u)})(1-e^{-k(T-v)})|u-v|^{2\beta
-2}dudv\\
&\leq\frac{1}{k^{2}}\int_{[0,T]^{2}}|u-v|^{2\beta-2}dudv+\int_{[0,T]^{2}}e^{-k(T-u)}e^{-k(T-v)}|u-v|^{2\beta-2}dudv\\
&\leq \frac{1}{k^{2}}[\frac{T^{2\beta }}{(2\beta -1)\beta}+ \frac{\Gamma(2\beta -1)}{k^{2\beta }}] \\
\end{aligned}
\end{equation}
因此, $\left\Vert f_{T}\right\Vert_{\mathfrak{H}}^{2}\leq C_{\beta}^{'}[T^{2\beta}]$,得出(\ref{33})中的结果。
\end{proof}
\end{proposition}
\begin{proposition}
假定 \ref{assumption:4.3.1} 成立, 且存在一个独立于 $T$ 的常数 $C>0$ ,那么对于任意的 $s,t\geq0$ 有,
\begin{equation}
E[|F_{t}-F_{s}|^{2}]\leq C_{\alpha ,\beta }|t-s|^{2\beta }
\end{equation}
\begin{proof}
首先,由不等式 (\ref{34}) 可以推出
\begin{equation}
E[|F_{t}-F_{s}|^{2}]\leq 2[E(|G_{t}-G_{s}|^2)+2E(|Z_{t}-Z_{s}|^{2})
\end{equation}
由引理\ref{tui2.1.2}, 可以得出 $\mathbb{E}(G_{t}-G_{s})^{2}\leq C_{\beta}|t-s|^{2\beta}$. 此外,我们有
\begin{equation}\label{317}
\begin{aligned}
E(|Z_{t}-Z_{s}|^{2})&=E[\int_{0}^{t}e^{-k(t-u)}dG_{u}-\int_{0}^{s}e^{-k(s-v)}dG_{v}]^{2}\\
&=E[e^{-kt}\int_{s}^{t}e^{ku}dG_{u}+(e^{-kt}-e^{-ks})\int_{0}^{s}e^{kv}dG_{v}]^{2}\\
&\leq E[e^{-kt}\int_{s}^{t}e^{ku}dG_{u}]^{2}+E[(e^{-k(t-s)}-1)e^{-ks}\int_{0}^{s}e^{kv}dG_{v}]^{2}
\end{aligned}
\end{equation}
由(\ref{317})中的第二项, 可以得到
\begin{equation}
E[(e^{-k(t-s)}-1)e^{-ks}\int_{0}^{s}e^{kv}dG_{v}]^{2}\leq C|t-s|^{2\beta}
\end{equation}
同时,可以得出
\begin{equation}
\begin{aligned}
E[e^{-kt}\int_{s}^{t}e^{ku)}dG_{u}]^{2}&\leq \int_{[s,t]^{2}}e^{-k(t-u)-k(t-v)}|u-v|^{2\beta-2}dudv+(\int_{s}^{t}e^{-k(t-u)u^{\beta-1}}du)^{2}\\
&\leq C^{'}|t-s|^{2\beta}
\end{aligned}
\end{equation}
因此,得出了预期的结果。
\end{proof}
\end{proposition}
\begin{proposition}\label{proposition326}
在假定1.1,和 $\gamma>\beta$下, 可以得到
$\underset{T\rightarrow \infty }{\lim }\frac{F_{T}}{T^\gamma}=0$ 几乎处处成立.
\begin{proof}
该证明类似于 \cite{Yong.C 2017}.
当 $\beta \in (\frac{1}{2},1)$时,由切比雪夫不等式和多重 Wiener-It$\hat{o}$ 积分的超收缩性可以推出,对于任意 $\varepsilon >0$ 和 $p(\gamma-\beta )>-1$,
\begin{equation}
\begin{aligned}
p(\frac{F_{n}}{n^\gamma}>\varepsilon )\leq \frac{EF_{n}^{p}}{n^{\gamma p}\varepsilon ^{p}}%
\leq \frac{C(EF_{n}^{2})^{p/2}}{n^{\gamma p}\varepsilon ^{p}}\leq \frac{C}{%
n^{p(\gamma-\beta )}}
\end{aligned}
\end{equation}
由 Borel-Cantelli 引理可以推出$\beta \in (\frac{1}{2},1)$,\begin{align}
 \lim_{n\rightarrow \infty }\frac{F_{n}}{n^{\gamma}}=0,a.s.
\end{align}
其次,存在两个独立于T的常数 $\alpha \in (0,1)$ 和 $C_{\alpha ,\beta }>0$ ,那么对于任意的 $|t-s|\leq 1$,\begin{align}
E[|F_{t}-F_{s}|^{2}]\leq C_{\alpha ,\beta }|t-s|^{2\beta }
\end{align}
然后, Garsia-Rumsey 不等式意味着对于任意实数 $p>\frac{4}{\alpha }$, $q>1$, 和整数
$n\geq 1$, \begin{align}
|F_{t}-F_{s}|\leq R_{p,q}n^{q/p},  \forall t,s\in[n,n+1]
\end{align}
其中 $R_{p,q}$ 是独立于n的随机常数。
最后,由于 $|\frac{F_{T}}{T^\gamma}|\leq \frac{1}{T^\gamma}|F_{T}-F_{n}|+\frac{n^{\gamma}}{T^\gamma}\frac{|F_{n}|}{n^{\gamma}}$.
其中 $n=[T]$ 是小于或等于实数T的最大整数,随着$T\rightarrow \infty $,可以得到 $\frac{F_{T}}{T^\gamma}$ 几乎处处收敛到0。
\end{proof}
\end{proposition}
\begin{proposition}
 $X_{T}$ 如 (\ref{23})表示, 那么当T趋于无穷时,
\begin{equation}\label{325}
\frac{1}{T}\int_{0}^{T}X_{t}^{2}dt%
{\rightarrow }C_{\beta }\Gamma (2\beta -1)k^{-2\beta }+\mu^2
\end{equation}
几乎处处成立。
\begin{proof}
根据(\ref{23})中 $X_{t}$ 的表达式, 可以得出
\begin{equation}\label{326}
\begin{aligned}
\frac{1}{T}\int_{0}^{T}X_{t}^{2}dt&=\frac{1}{T}\int_{0}^{T}[\mu
(1-e^{-kt})+\int_{0}^{t}e^{-k(t-s)}dG_{s}]^2dt\\
&=\frac{1}{T}\int_{0}^{T}[\mu (1-e^{-kt})]^{2}dt+\frac{1}{T}%
\int_{0}^{T}[\int_{0}^{t}e^{-k(t-s)}dG_{s}]^{2}dt\\
&+\frac{2}{T}\int_{0}^{T}[\mu (1-e^{-kt})\int_{0}^{t}e^{-k(t-s)}dG_{s}]dt\\
&=:I_{1}+I_{2}+I_{3}\\
\end{aligned}
\end{equation}
对于 $I_{1}$ 这一项, 易得出
\begin{equation}\label{327}
I_{1}=\frac{1}{T}\int_{0}^{T}[\mu (1-e^{-kt})]^{2}dt\overset{a.s}{\rightarrow }\mu ^{2}
\end{equation}
用在 \cite{Yong.C 2020}中相同的方法, 可以求得
\begin{equation}\label{328}
I_{2}=\frac{1}{T}\int_{0}^{T}[\int_{0}^{t}e^{-k(t-s)}dG_{s}]^{2}dt\overset{a.s}%
{\rightarrow }C_{\beta }\Gamma (2\beta -1)k^{-2\beta }
\end{equation}
可以推断出
\begin{equation}\label{329}
I_{3}=\frac{2\mu }{T}\int_{0}^{T}(\int_{0}^{t}e^{-k(t-s)}dG_{s})dt-\frac{2\mu
}{T}\int_{0}^{T}e^{-kt}(\int_{0}^{t}e^{-k(t-s)}dG_{s})dt
\end{equation}
这里由计算可得
\begin{equation}\label{330}
\begin{aligned}
\frac{2\mu }{T}\int_{0}^{T}e^{-kt}(\int_{0}^{t}e^{-k(t-s)}dG_{s})dt
&=\frac{2\mu }{T}\int_{0}^{T}dG_{s}\int_{s}^{T}e^{-k(2t-s)}dt\\
&=\frac{2\mu }{T}\int_{0}^{T}\frac{1}{2k}(e^{-ks}-e^{-k(2T-s)})dG_{s}\\
&=\frac{\mu }{T}\int_{0}^{T}\frac{1}{k}e^{-ks}dG_{s}
-\frac{\mu }{T}\int_{0}^{T}\frac{1}{k}e^{-k(2T-s)}dG_{s}\\
\end{aligned}
\end{equation}
对于(\ref{330})中的第一项, 令 $\frac{1}{T}\int_{0}^{T}e^{-ks}dG_{s}=M_{T}$, 可以得出
\begin{equation}
\begin{aligned}
E[M_{T}^{2}]&=\int_{[0,T]^{2}}e^{-ks}\cdot e^{-kr}\cdot \frac{\partial
^{2}R(s,r)}{\partial s\partial r}dsdr\\
&\leq \int_{[0,T]^{2}}e^{-ks}\cdot e^{-kr}|s-r|^{2\beta
-2}dsdr+(\int_{0}^{T}e^{-ks}s^{\beta -1}ds)^{2}\leq C_{\beta }T^{2\beta }\\
\end{aligned}
\end{equation}
当 $\beta \in (\frac{1}{2},\frac{3}{4})$ 和 $p(\beta -1)<-1$ 时,
\begin{equation}
\begin{aligned}
P(|\frac{M_{n}}{n}|>\varepsilon )&\leq \frac{E|M_{n}|^{p}}{n^{p}\varepsilon^{p}}
&\leq \frac{E(|M_{n}^{2}|)^{\frac{p}{2}}}{n^{p}\varepsilon ^{p}}
&\leq n^{p(\beta -1)}
\end{aligned}
\end{equation}
可以推出$\underset{n\rightarrow \infty }{\lim }\frac{M_{n}}{n}=0,a.s$.\\
此外,存在一个独立于T 的常数 $C_{\beta }>0$,那么,对于任意 $|t-s|\leq 1$,
\begin{equation}
  \begin{split}
    E[|M_{t}-M_{s}|^{2}]&=\int_{[s,t]^2}e^{-ks}e^{-kr}\frac{\partial^2 R(m,m)}{\partial m,\partial n}dmdn\\
    &\leq \int_{[s,t]^2}\frac{\partial^2 R(m,n)}{\partial m,\partial n}dmdn\\
    &\leq E[|G_{t}-G_{s}|^{2}]\leq C_{\beta }(t-s)^{2\beta }
  \end{split}
\end{equation}
那么由 Garsia-Rumsey 不等式可推出对于任意的 $p>\frac{4}{\alpha }$, $q>1$和整数
$n\geq 1$, \begin{align}
|M_{t}-M_{s}|\leq R_{p,q}n^{q/p},  \forall t,s\in[n,n+1]
\end{align}
其中 $R_{p,q}$ 是独立于n 的随机常数。
最后,推出 $|\frac{M_{T}}{T}|\leq \frac{1}{T}|M_{T}-M_{n}|+\frac{n}{T}\frac{|M_{n}|}{n}$.
其中 $n=[T]$ 是小于或等于实数T的最大整数, 当 $T\rightarrow \infty $ 时,可以推出 $\frac{M_{T}}{T} $ 几乎必然收敛到0。
\begin{equation}\label{331}
\frac{\mu }{T}\int_{0}^{T}\frac{1}{k}e^{-ks}dG_{s}\overset{a.s}{\rightarrow }0
\end{equation}
对于(\ref{330})中的最后一项, 可以得出
\begin{equation}\label{332}
\frac{\mu }{T}\int_{0}^{T}\frac{1}{k}e^{-k(2T-s)}dG_{s}=e^{-kT}\frac{\mu }{T}\int_{0}^{T}\frac{1}{k}e^{-k(T-s)}dG_{s}\overset{a.s}{\rightarrow }0
\end{equation}
最后一步参照\cite{Yong.C 2020}。\\
结合以上结果,可以得出
\begin{equation}\label{333}
\frac{2\mu }{T}\int_{0}^{T}e^{-kt}(\int_{0}^{t}e^{-k(t-s)}dG_{s})dt\overset{a.s}{\rightarrow }0
\end{equation}
这可以推出
\begin{equation}\label{334}
I_{3}=\frac{2}{T}\int_{0}^{T}[\mu (1-e^{-kt})\int_{0}^{t}e^{-k(t-s)}dG_{s}]dt\overset{a.s}{\rightarrow }0
\end{equation}
由(\ref{326})-(\ref{334})可得, 当 T 趋于无穷时,可以得出
\begin{equation}
(\frac{1}{T}\int_{0}^{T}X_{t}^{2}dt)\overset{a.s}{\rightarrow }C_{\beta }\Gamma (2\beta -1)k^{-2\beta }+\mu^{2}
\end{equation}
因此, $\widehat{k}$的二阶矩估计量是强相合的,即, $\widehat{k}\overset{a.s}{\rightarrow }k$。
\end{proof}
\end{proposition}
\subsection{最小二乘估计量的相合性}
\label{subsection:3.2.2}
\vspace{0.5cm}
\label{subsection:3.1.2}
\begin{proposition}\label{key}
  如 (\ref{23})中所表示的 $(X_{t},t\in[0,T])$, 那么
  \begin{equation}\label{336}
  \frac{1}{T}\int_{0}^{T}X_{t}dX_{t}\overset{a.s}{\rightarrow } C_{\gamma},
  \end{equation}
 随着T趋于无穷,其中 $C_{\gamma}$ 表示合适的正常数。
  \begin{proof}
  根据 (\ref{model}), 我们将随即积分 $\frac{1}{T} X_tdX_t$ 可以表示为
  \begin{equation}
  \frac{1}{T}\int_{0}^{T}X_{t}dX_{t}=\frac{k\mu}{T}\int_{0}^{T}X_{t}dt-\frac{k}{T}\int_{0}^{T}X_{t}^{2}dt+\frac{1}{T}\int_{0}^{T}X_{t}dG_{t}
  \end{equation}
  由 (\ref{23})可得出
  \begin{equation}\label{338}
  \frac{1}{T}\int_{0}^{T}X_{t}dG_{t}=\frac{1}{T}\int_{0}^{T}\mu(1-e^{-kt})dG_{t}+\frac{1}{T}\int_{0}^{T}\int_{0}^{T}e^{-k(t-s)}dG_{s}dG_{t}
  \end{equation}
  对于 (\ref{338})中的第一项, 由(\ref{331})可得
  \begin{equation}\label{339}
  \frac{1}{T}\int_{0}^{T}\mu(1-e^{-kt})dG_{t}=\frac{\mu}{T}\int_{0}^{T}dG_{t}-\frac{1}{T}\int_{0}^{T}e^{-kt}dG_{t}\overset{a.s}{\rightarrow }0
  \end{equation}
   从\cite{Yong.C 2020}的proposition3.7可知, 随着T趋于无穷, $\frac{1}{T}\int_{0}^{T}\int_{0}^{T}e^{-k(t-s)}dG_{s}dG_{t}$ 几乎必然收敛到0。
  然后,结合 (\ref{338}) 和 (\ref{339}),足以证明
  \begin{equation}\label{340}
  \frac{1}{T}\int_{0}^{T}X_{t}dG_{t}\overset{a.s}{\rightarrow }0
  \end{equation}
  同时,由 proposition3.6可知
  \begin{equation}\label{341}
  \frac{k\mu}{T}\int_{0}^{T}X_{t}dt-\frac{k}{T}\int_{0}^{T}X_{t}^{2}dt\overset{a.s}{\rightarrow }C_{\gamma}
  \end{equation}
  因此,我们可以得出 (\ref{336})中的结果。
  \end{proof}
  \end{proposition}
  \begin{proposition}
  当$\beta \in [\frac{1}{2},1)$, 有 $\hat{\mu} _{LS}\overset{a.s}{\rightarrow}\mu $ 。
  \begin{proof}
  首先,将原式带入并展开
  \begin{equation}\label{342}
  \widehat{\mu }_{LS}-\mu =\frac{\frac{X_{T}}{T}\cdot
  \frac{1}{T}\int_{0}^{T}X_{t}^{2}dt-\frac{1}{T}\int_{0}^{T}X_{t}dX_{t}\cdot
  \frac{1}{T}\int_{0}^{T}X_{t}dt-\mu (\frac{X_{T}}{T}\cdot
  \frac{1}{T}\int_{0}^{T}X_{t}dt-\frac{1}{T}\int_{0}^{T}X_{t}dX_{t})}{%
  \frac{X_{T}}{T}\cdot \frac{1}{T}\int_{0}^{T}X_{t}dt-\frac{1}{T}%
  \int_{0}^{T}X_{t}dX_{t}}
  \end{equation}
  如下表示partI,根据命题 \ref{key}, 可以推出
  \begin{equation}\label{343}
  partI=\frac{1}{T}\int_{0}^{T}X_{t}dX_{t}[\frac{1}{T}\int_{0}^{T}X_{t}dt-\mu ]\overset{a.s}{\rightarrow }0
  \end{equation}
  如下表示 partII , 由proposition3.6可知
  \begin{equation}\label{344}
  partII=\frac{X_{T}}{T}(\frac{1}{T}\int_{0}^{T}X_{t}^{2}dt-\frac{\mu}{T}\int_{0}^{T}X_{t}dt)\overset{a.s}{\rightarrow }0
  \end{equation}
  最后,即为所证。
  \end{proof}
  \end{proposition}
  \begin{proposition}
  当 $\beta\in[\frac{1}{2},1)$,那么有 $\hat{k}_{LS}\overset{a.s}{\rightarrow }k$.
  \begin{proof}
  由 (\ref{109})中的表达式$\hat{k}_{LS}$可得
  \begin{equation}\label{345}
  \widehat{k}_{LS}-k
  =\frac{\frac{1}{T^{2}}X_{T}\int_{0}^{T}X_{t}dt-k\mu \frac{1}{T}%
  \int_{0}^{T}X_{t}dt-\frac{1}{T}\int_{0}^{T}X_{t}dG_{t}+k(\frac{1}{T}%
  \int_{0}^{T}X_{t}dt)^{2}}{\frac{1}{T}\int_{0}^{T}X_{t}^{2}dt-(\frac{1}{T}%
  \int_{0}^{T}X_{t}dt)^{2}}
  \end{equation}
  首先,如下表示 partA ,
  \begin{equation}\label{346}
  partA=k(\frac{1}{T}\int_{0}^{T}X_{t}dt)^{2}-k\mu \frac{1}{T}\int_{0}^{T}X_{t}dt\overset{a.s}{\rightarrow }0
  \end{equation}
  接着, 如下表示partB,
  \begin{equation}\label{34.}
  \begin{aligned}
  partB=\frac{1}{T^{2}}X_{T}\int_{0}^{T}X_{t}dt&=\frac{X_{T}}{T}\frac{1}{T}\int_{0}^{T}X_{t}dt
  \\
  &=[\frac{1}{T}\mu(1-e^{-kT})+\frac{1}{T}\int_{0}^{T}e^{-k(t-s)}dG_{s}]\cdot\frac{1}{T}\int_{0}^{T}X_{t}dt
  \end{aligned}
  \end{equation}
  由 \cite{Yong.C 2020} 中的proposition3.7 和 (\ref{24})可得出,
  \begin{equation}\label{347}
  \frac{1}{T^{2}}X_{T}\int_{0}^{T}X_{t}dt\overset{a.s}{\rightarrow }0
  \end{equation}
  结合 (\ref{340}), (\ref{346}) 和 (\ref{347}),证得命题。
  \end{proof}
  \end{proposition}
\section{估计量的渐近正态性}
\label{section:3.3}
\subsection{矩估计量的渐近正态性}
\label{subsection:3.3.1}
我们需要几个引理,提供充分的条件来证明 $\widehat{\mu}$的渐近正态性。
\begin{lemma}\label{yin3.3.1}
当$\beta \in (\frac{1}{2},1)$,且T趋于无穷时,
\begin{equation}\label{400}
T^{\beta-\frac{1}{2}}(\frac{1}{T}\int_{0}^{T}X_{t}dt-\mu )\to 0
\end{equation}
almost surely .
\begin{proof}
\begin{equation}
\begin{aligned}
T^{\beta -\frac{1}{2}}(\frac{1}{T}\int_{0}^{T}X_{t}dt-\mu )
&=T^{\beta -\frac{1}{2}}(\frac{1}{T}\int_{0}^{T}\mu (1-e^{-kt})dt+\frac{1}{T}%
\int_{0}^{T}\int_{0}^{t}e^{-k(t-s)}dG_{s}dt-\mu )\\
&=T^{\beta -\frac{1}{2}}(\frac{1}{T}\int_{0}^{T}\int_{0}^{t}e^{-k(t-s)}dG_{s}dt)\\
&=\frac{1}{T^{\frac{3}{2}-\beta }}\int_{0}^{T}\int_{0}^{t}e^{-k(t-s)}dG_{s}dt\\
&=\frac{k}{T^{\frac{3}{2}-\beta }}\int_{0}^{T}(1-e^{-k(T-s)})dG_{s}\\
&=\frac{k}{T^{\frac{3}{2}-\beta }}\cdot G_{T}-\frac{k\left\Vert f_{T}\right\Vert}{T^{\frac{3}{%
2}-\beta }}\frac{I_{1}(f_{T})}{\left\Vert f_{T}\right\Vert}\\
\end{aligned}
\end{equation}
其中 $\frac{3}{2}-\beta >\beta $,$\frac{G_{T}}{T^{\beta }}$ 和 $\frac{I_{1}(f_{T})}{||f_{T}||}$ 也是标准正态分布得随机变量,与命题3.5一起证明了引理的成立。
\end{proof}
\end{lemma}
\begin{proposition}
对于 $\beta\in[\frac{1}{2},1)$, 可以得出由 (\ref{4})中定义的 $\hat\mu$。
\begin{equation}\label{402}
T^{1-\beta }(\hat{\mu }-\mu )\overset{law}{\to }N(0,\frac{1}{k^{2}})
\end{equation}
\begin{proof}
首先可以得出
\begin{equation}\label{403}
\begin{aligned}
T^{1-\beta }(\widehat{\mu}-\mu )
&=T^{1-\beta }[\frac{1}{T}\int_{0}^{T}\mu (1-e^{-kt})dt+\frac{1}{T}%
\int_{0}^{T}\int_{0}^{t}e^{-k(t-s)})dG_{s}dt-\mu ]\\
&=\frac{1}{T^{\beta }}\int_{0}^{T}\int_{0}^{t}e^{-k(t-s)})dG_{s}dt\\
&=\frac{G_{T}}{kT^{\beta }}-\frac{1}{kT^{\beta }}\int_{0}^{T}e^{-k(T-s)})dG_{s}\\
\end{aligned}
\end{equation}
由 \cite{Yong.C 2020}中的Proposition3.8可得 $ \frac{1}{kT^{\beta }}\int_{0}^{T}e^{-k(T-s)})dG_{s}\rightarrow 0$,
$\frac{G_{T}}{T^{\beta }}$ 是标准正态分布。最后,通过 Slutsky定理,得出  (\ref{402}) 的渐近正态成立。
\end{proof}
\end{proposition}
\begin{proposition}
记一个依赖于 $k$ 和 $\beta$ 的常数为 $a:=C_{\beta}\Gamma(2\beta-1)k^{-2\beta}$,
那么对于 $\beta\in[\frac{1}{2},\frac{3}{4})$ 和 $T\to\infty$,有
\begin{equation}
\sqrt{T}(\hat{k}-k){\rightarrow}N(0,a^{2}\sigma_{\beta}^{2}/4\beta^{2})
\end{equation}
其中$\sigma _{\beta }^{2}=(4\beta -1)[1+\frac{\Gamma (3-4\beta )\Gamma (4\beta-1)}{\Gamma (2\beta )\Gamma (2-2\beta )}]$。
\begin{proof}
首先,可得
\begin{small}
\begin{equation}
\begin{aligned}
\sqrt{T}(\frac{1}{T}\int_{0}^{T}X_{t}^{2}dt-(\frac{1}{T}\int_{0}^{T}X_{t}dt)^{2}&-a)
=\sqrt{T}\biggl(\frac{1}{T}\int_{0}^{T}[\mu(1-e^{-kt})]^{2}dt
+\frac{1}{T}\int_{0}^{T}[\int_{0}^{t}e^{-k(t-s)}dG_{s}]^{2}dt\\
&+\frac{2}{T}\int_{0}^{T}[\mu(1-e^{-kt})\int_{0}^{t}e^{-k(t-s)}dG_{s}]dt-(\frac{1}{T}\int_{0}^{T}X_{t}dt)^{2}-a\biggr)\\
\end{aligned}
\end{equation}
\end{small}
事实上,这里有
\begin{equation}
\frac{1}{T}\int_{0}^{T}[\mu(1-e^{-kt})]^{2}dt-(\frac{1}{T}\int_{0}^{T}X_{t}dt)^{2}\overset{a.s}\to 0.
\end{equation}
同时,由 (\ref{331})可得
\begin{equation}
\frac{\mu }{\sqrt{T}}\int_{0}^{T}\frac{1}{k}e^{-ks}dG_{s}\overset{a.s}\to 0
\end{equation}
由 \cite{Yong.C 2020}中Proposition3.8可得,
\begin{equation}
\frac{\mu }{\sqrt{T}}\int_{0}^{T}\frac{1}{k}e^{-k(2T-s)}dG_{s}\overset{a.s}\to 0
\end{equation}
由这两个结果可得出
\begin{equation}
\sqrt{T}(\frac{1}{T}\int_{0}^{T}X_{t}^{2}dt-(\frac{1}{T}%
\int_{0}^{T}X_{t}dt)^{2}-a)\overset{law}{\rightarrow }N(0,a^{2}\sigma_{\beta }^{2}/k)
\end{equation}
最后,由delta方法可得出渐近正态成立。
\end{proof}
\end{proposition}
\subsection{最小二乘估计量的渐近正态性}
\label{subsection:3.3.2}
这一小节用来讨论  $\hat{\mu}_{LS}$ 和$\hat{k}_{LS}$的渐近正态性。
\begin{proposition}
对于 $\beta \in [\frac{1}{2},1)$, 且$T\to \infty$,
\begin{equation}
T^{1-\beta }(\widehat{\mu}_{LS}-\mu )\overset{law}{\to }N(0,\frac{1}{k^{2}}).
\end{equation}
\begin{proof}
通过 (\ref{342})的表示,
\begin{equation}
T^{1-\beta }(\widehat{\mu }_{LS}-\mu )=\frac{T^{1-\beta}(partI + partII)}{%
\frac{X_{T}}{T}\cdot \frac{1}{T}\int_{0}^{T}X_{t}dt-\frac{1}{T}%
\int_{0}^{T}X_{t}dX_{t}}
\end{equation}
首先, 结合 (\ref{336}) 和 Proposition4.1, $T^{1-\beta }\cdot partI$ 可写作
\begin{equation}
T^{1-\beta }\cdot \frac{1}{T}\int_{0}^{T}X_{t}dX_{t}[\frac{1}{T}%
\int_{0}^{T}X_{t}dt-\mu ]\overset{law}{\to }N(0,\frac{1}{k^{2}})
\end{equation}
使用类似于 $\widehat{\mu}$强收敛的方法,可得
\begin{equation}
T^{1-\beta }\cdot partII \overset{a.s}\to 0
\end{equation}
那么,使用 Slutsky's 定理,得出结论。
\end{proof}
\end{proposition}
\begin{proposition}
当$k>0$和 $\beta \in (\frac{1}{2},\frac{3}{4})$时, 那么以下收敛结果成立
\begin{equation}
\sqrt{T}(\widehat{k}_{LS}-k)\overset{law}{\rightarrow }N(0,4ka^{2}\sigma _{\beta }^{2}).
\end{equation}
\begin{proof}
由 (\ref{345})可得
\begin{equation}
\sqrt{T}(\widehat{k}_{LS}-k)
=\frac{\biggl(\sqrt{T}\frac{1}{T^{2}}X_{T}\int_{0}^{T}X_{t}dt-k\mu \frac{1}{T}%
\int_{0}^{T}X_{t}dt-\frac{1}{T}\int_{0}^{T}X_{t}dG_{t}+k(\frac{1}{T}%
\int_{0}^{T}X_{t}dt)^{2}\biggr)}{\frac{1}{T}\int_{0}^{T}X_{t}^{2}dt-(\frac{1}{T}%
\int_{0}^{T}X_{t}dt)^{2}}
\end{equation}
w首先只考虑其中的两项,即  $\sqrt{T}\cdot partA$ 可被写为
\begin{equation}
\begin{aligned}
\sqrt{T}[k(\frac{1}{T}\int_{0}^{T}X_{t}dt)^{2}-\frac{k\mu }{T}%
\int_{0}^{T}X_{t}dt]&=\sqrt{T}[(\frac{k}{T}\int_{0}^{T}X_{t}dt-k\mu )\frac{1}{%
T}\int_{0}^{T}X_{t}dt]\\
&=\sqrt{T}[(\frac{k}{T}\int_{0}^{T}\mu (1-e^{-kt})dt\\
&+\frac{k}{T}\int_{0}^{T}\int_{0}^{t}e^{-k(t-s)}dG_{s}dt-k\mu )\frac{1}{T}%
\int_{0}^{T}X_{t}dt]\\
&=[\frac{k\mu }{\sqrt{T}}\int_{0}^{T}-e^{-kt}dt+\frac{k}{\sqrt{T}}%
\int_{0}^{T}\int_{0}^{t}e^{-k(t-s)}dG_{s}dt]\frac{1}{T}\int_{0}^{T}X_{t}dt\\
\end{aligned}
\end{equation}
其中 $\frac{k\mu }{\sqrt{T}}\int_{0}^{T}-e^{-kt}dt\overset{a.s}{\rightarrow }0$,可以推出
\begin{equation}
\begin{aligned}
\frac{k}{\sqrt{T}}\int_{0}^{T}\int_{0}^{t}e^{-k(t-s)}dG_{s}dt&=\frac{k}{%
\sqrt{T}}\int_{0}^{T}dG_{s}\int_{s}^{T}e^{-k(t-s)}dt=\frac{1}{\sqrt{T}}%
\int_{0}^{T}dG_{s}-\frac{1}{\sqrt{T}}\int_{0}^{T}e^{-k(T-s)}dG_{s}\\
&=\frac{G_{T}}{\sqrt{T}}-\frac{1}{\sqrt{T}}e^{-kT}\int_{0}^{T}e^{ks}dG_{s}\\
\end{aligned}
\end{equation}
由Proposition3.1可得 $\frac{1}{\sqrt{T}}e^{-kT}\int_{0}^{T}e^{ks}dG_{s}\overset{a.s}{\rightarrow }0$, 因此可推出
\begin{equation}
\begin{aligned}
\sqrt{T}(\widehat{k}_{LS}-k)&=-\frac{\frac{1}{\sqrt{T}}%
\int_{0}^{T}X_{t}dG_{t}+(\frac{X_{T}}{\sqrt{T}}+\frac{G_{T}}{\sqrt{T}})\frac{%
1}{T}\int_{0}^{T}X_{t}dt}{\frac{1}{T}\int_{0}^{T}X_{t}^{2}dt-(\frac{1}{T}%
\int_{0}^{T}X_{t}dt)^{2}}\\
&=-\frac{\frac{1}{\sqrt{T}}\int_{0}^{T}\mu (1-e^{-kt})dG_{t}+\frac{1}{\sqrt{T%
}}\int_{0}^{T}\int_{0}^{t}e^{-k(t-s)}dG_{t}dG_{s}}{\frac{1}{T}%
\int_{0}^{T}X_{t}^{2}dt-(\frac{1}{T}\int_{0}^{T}X_{t}dt)^{2}}+\frac{(\frac{%
X_{T}}{\sqrt{T}}+\frac{G_{T}}{\sqrt{T}})\frac{1}{T}\int_{0}^{T}X_{t}dt}{%
\frac{1}{T}\int_{0}^{T}X_{t}^{2}dt-(\frac{1}{T}\int_{0}^{T}X_{t}dt)^{2}}\\
&=-\frac{\mu }{\sqrt{T}}G_{T}+\frac{\mu }{\sqrt{T}}\int_{0}^{T}e^{-kt}dG_{t}-%
\frac{1}{\sqrt{T}}\int_{0}^{T}\int_{0}^{t}e^{-k(t-s)}dG_{t}dG_{s}+\frac{G_{T}%
}{\sqrt{T}}\frac{1}{T}\int_{0}^{T}X_{t}dt\\
&=\frac{G_{T}}{\sqrt{T}}(\frac{1}{T}\int_{0}^{T}X_{t}dt-\mu )+\frac{\mu }{%
\sqrt{T}}\int_{0}^{T}e^{-kt}dG_{t}-\frac{1}{\sqrt{T}}\int_{0}^{T}%
\int_{0}^{t}e^{-k(t-s)}dG_{t}dG_{s}\\
\end{aligned}
\end{equation}
很显然, $\frac{\mu }{\sqrt{T}}\int_{0}^{T}e^{-kt}dG_{t}\overset{a.s}{\rightarrow }0$, 结合\ref{yin3.3.1}, 可推出$\frac{G_{T}}{\sqrt{T}}(\frac{1}{T}\int_{0}^{T}X_{t}dt-\mu
)\overset{a.s}{\rightarrow }0$。因此,我们可得出
\begin{equation}
\begin{aligned}
\sqrt{T}(\widehat{k}_{LS}-k)&=\frac{-\frac{1}{\sqrt{T}}\int_{0}^{T}%
\int_{0}^{t}e^{-k(t-s)}dG_{t}dG_{s}}{\frac{1}{T}\int_{0}^{T}X_{t}^{2}dt-(%
\frac{1}{T}\int_{0}^{T}X_{t}dt)^{2}}&=\frac{-\frac{1}{\sqrt{T}}%
I_{2}(e^{-k(t-\cdot )})}{\frac{1}{T}\int_{0}^{T}X_{t}^{2}dt-(\frac{1}{T}%
\int_{0}^{T}X_{t}dt)^{2}}\\
\end{aligned}
\end{equation}
其中 $\sigma _{\beta }^{2}=(4\beta -1)[1+\frac{\Gamma (3-4\beta )\Gamma (4\beta-1)}{\Gamma (2\beta )\Gamma (2-2\beta )}]$. \par
由 \cite{Yong.C 2020}中(4.8) 可得, $-\frac{1}{\sqrt{T}}%
I_{2}(e^{-k(t-\cdot )}) \overset{law}\to N(0,4ka^2\sigma_{\beta}^{2})$. 那么,  由Slutsky定理可以推出 (\ref{325}),的渐近正态时成立的。
\end{proof}
\end{proposition}
\chapter{在一般高斯过程驱动下的带周期均值的O-U过程}
\label{chapter:4}
 当O-U过程中的布朗运动被一般高斯过程代替,漂移系数变成带周期均值的函数 $L(t)=\sum_{i=1}^{p}\mu_{i}\varphi_{i}(t)$ 得出具体的模型如下:
\begin{equation}
\label{equation:3.1}
  dX_{t}=k(\sum_{i=1}^{p}\mu_{i}\varphi_{i}(t)-X_{t})dt+\sigma dG_{t},  X_0=0
\end{equation}
其中 $\varphi_{i}(t),i=1,...,p$ 是有界且周期为1的函数,实数 $\mu_{i},i=1,...,p$ 为未知参数,假定参数 $\sigma=1$,而 $G_{t}$ 满足如下假定。
\begin{Assumption}
对于$\beta \in (\frac{1}{2},1)$, 任意 $t\neq s\in \lbrack 0,\infty )$ 的协方差函数为 $R(t,s)=E[G_{t}G_{s}]$ 。
\begin{align}
    \frac{\partial ^{2}}{\partial t\partial s}R(t,s)&=C_{\beta }|t-s|^{2\beta-2}+\Psi (t,s)\end{align}
with  \begin{align}
    |\Psi (t,s)|&\leq C_{\beta }^{^{\prime }}|ts|^{\beta -1}
\end{align}
其中常数 $\beta ,C_{\beta }>0,C_{\beta }^{^{\prime }}\geq 0$ 不依赖于T。此外,对于任意 $t\geq 0$, 有$R(0,t)=0$。
\end{Assumption}

则模型的平稳解为:
\begin{equation}\label{40.2}
  X_{t}=\int_{0}^{t}e^{-k(t-s)}L(s)ds+\int_{0}^{t}e^{-k(t-s)}dG_{t}
\end{equation}
为了简化之后的计算,可以将平稳解写成下列形式:
\begin{equation}
  X_t = h(t) + Z_{t}
\end{equation}
其中 $h(t):=e^{-kt}\sum_{i=1}^{p}\mu_{i}\int_{0}^{t}e^{ks}\varphi_{i}(s)ds$,$z(t):=e^{-kt}\int_{0}^{t}e^{ks}dG_{s}$。

这里引用Franke and Kott(2013)相同的方法构造估计量\cite{Franke B 2013},使用最小二乘方法估计参数 $\theta$。 首先,我们考虑下列随机微分方程的 $p+1$维的参数向量的估计问题
\begin{equation}
  dX_{t}=\theta f(t,X_{t})dt+\sigma dG_{t}
\end{equation}
其中 $f(t,x)=(f_{1}(t,x),...,f_{p+1}(t,x))$ 是合适的实值函数。在时间间隔为 $[0,T]$ 里对上述方程离散化即 $\Delta t:=T/N, i=1,...,N$
\begin{equation}
  X_{(i+1)\Delta t} -X_{i\Delta t} =\sum_{j=1}^{p+1}f_{j}(i\Delta,X_{i\Delta t})\theta_{j}\Delta t+\sigma(G_{(i+1)\Delta t}-G_{i\Delta t})
\end{equation}
根据最小二乘方法估计线性方程的参数,我们最小化下列最小二乘函数
\begin{equation}
  \mathcal{G} :(\theta_1,...,\theta_{p+1}) \mapsto \sum_{i=1}^{N}(X_{(i+1)\Delta t}-X_{i\Delta t}-\sum_{j=1}^{p+1}f_{j}(i\Delta t,X_{i\Delta t})\theta_j\Delta t)^2
\end{equation}
上述方程中最小化 $\widetilde \theta_{T,\Delta t}$ 得出如下形式\cite{Franke B 2013}:
\begin{equation}
  \widetilde \theta_{T,\Delta t}=Q_{T,\Delta t}^{-1}P_{T,\Delta t}
\end{equation}
这使我们得到在连续时间上的估计量 $\hat \theta_{T}=Q_{T}^{-1}P_{T}$,其中
$$
Q_{T}=
\begin{pmatrix}
  \int_0^T f_1(t,X_t)f_1(t,X_t)dt & \cdots & \int_0^T f_1(t,X_t)f_{p+1}(t,X_t)dt   \\
    \vdots                                                                 & \vdots  \\
    \int_0^T f_{p+1}(t,X_t)f_1(t,X_t)dt & \cdots  & \int_0^T f_{p+1}(t,X_t)f_{p+1}(t,X_t)dt  \\
\end{pmatrix}
$$
\begin{equation}
  P_{T}:=(\int_0^T f_1(t,X_t)dX_t,...,\int_0^T f_p(t,X_t)dX_t)^t
\end{equation}
在一般高斯噪声过程驱动下的带周期均值的O-U过程中,我们有 $\theta=(\mu_1,...,\mu_p,\alpha)$ 和 $f(t,x):=(\varphi_1,...,\varphi_p,-x)^t$, 对于 $T=n\nu$,则 $\theta$ 估计量为
\begin{equation}
  \hat\theta_n :=Q_n^{-1}P_{n}
\end{equation}
其中,
\begin{equation}
  P_n:=(\int_0^{n\nu} \varphi_1(t)dX_t,...,\int_0^{n\nu} \varphi_p(t)dX_t,-\int_0^{n\nu} X_tdX_t)^t
\end{equation}
$$
Q_{n}=
\begin{pmatrix}
  G_n    &  -a_n \\
  -a_n^t &  b_n  \\
\end{pmatrix}
$$
这里有
$$
G_{n}=
\begin{pmatrix}
  \int_0^{n\nu} \varphi_1(t)\varphi_1(t)dt & \cdots & \int_0^{n\nu} \varphi_1(t)\varphi_p(t)dt \\
  \vdots                                                                 & \vdots          \\
  \int_0^{n\nu} \varphi_p(t)\varphi_1(t)dt & \cdots & \int_0^{n\nu} \varphi_p(t)\varphi_p(t)dt \\
\end{pmatrix}
$$
\begin{equation}
  a_n^t:=(\int_0^{n\nu} \varphi_1(t)X_tdt,...,\int_0^{n\nu} \varphi_p(t)X_tdt)
\end{equation}
\begin{equation}
  b_n :=\int_0^{n\nu} X_t^2dt
\end{equation}
为了简化计算,我们假定周期 $\nu=1$, 因此得到估计量 $\hat\theta_n$,
\begin{proposition}(\cite{Dehling 2016})
  \begin{equation}
    \hat\theta_n = \theta + Q_{n}^{-1}R_n
  \end{equation}
其中,
\begin{equation}
  R_n :=(\int_0^t \varphi_1(t)dG_t,...,\int_0^t \varphi_p(t)dG_t,-\int_0^n X_tdG_t)^t
\end{equation}
\end{proposition}

\begin{proposition}(\cite{Dehling 2016})
  $$
  Q_{n}^{-1}=\frac{1}{n}
  \begin{pmatrix}
    I_p + \gamma_n \Lambda_n \Lambda_n^t & \gamma_n \Lambda_n \\
    \gamma_n \Lambda_n^t                 &  \gamma_n          \\
  \end{pmatrix}
  $$
其中,$\gamma_n=(\gamma_{n,1},...,\gamma_{n,p})^t:=(\frac{1}{n}\int_0^n \varphi_1(t)X_tdt,...,\frac{1}{n}\int_0^n \varphi_p(t)X_tdt)^t$, $\gamma_n:=(\frac{1}{n}\int_{0}^{n}X_t^2-\sum_1^{p}\Lambda_{n,i}^2)^{-1}$
\end{proposition}
\section{主要结果}
\label{section:4.1}

\section{最小二乘估计量的相合性}
\label{section:3.2.1}
\begin{theorem}
  对于任意的 $\beta\in(\frac{1}{2},1)$, 当$n\rightarrow\infty$时,有$\hat\theta_n$是一致收敛到 $\theta$。
\begin{proof}
  根据命题4.2.1,可以得到 $\hat\theta_n=\theta + nQ_n^{-1}\cdot\frac{1}{n}R_n$,可直接由下列两个命题(命题4.2.1和命题4.2.2)得证。
\end{proof}
\end{theorem}
\begin{proposition}
  对于 $\beta\in(\frac{1}{2},1)$,当$n\rightarrow\infty$时,可以得到 $\frac{1}{n}R_n$ 是几乎处处收敛到0的。
\begin{proof}
  用\cite{Chen 2017}中相似的证明方法。根据假定$\varphi_i(t),i=1,2,...,p$,则有$sup_{t}\vert\varphi_i(t)\vert\leq C < \infty$,令$H_{n}=\int_0^{n}\varphi_{i}(t)dG_t$,那么首先计算$p+1$ 维列向量 $R_n$ 的前 $p$ 项元素$H_{n}$ 和增量$H_{t}-H_{s}$ 的二阶矩估计为
  \begin{equation}
  \begin{aligned}
    \mathbb{E}[H_{n}^2] &\leq C_{\beta}\int_{0}^{n}\int_{0}^{n}\varphi_{i}(u)\varphi_{i}(v)\vert u-v\vert^{2\beta-2} dudv \\
    &+ C_{\alpha,\beta}\int_{0}^{n}\int_{0}^{n}\varphi_{i}(u)\varphi_{i}(v)\vert uv \vert^{\beta-1}dudv \\
     & \leq C_{\beta}\int_{0}^{n}\int_{0}^{n} \vert u-v \vert ^{2\beta-2}dudv + C_{\alpha,\beta}\int_{0}^{n}\int_{0}^{n} \vert uv \vert^{\beta-1}dudv \\
     & \leq \frac{C_{\beta} n^{2\beta}}{\beta(2\beta-1)} + C_{\alpha,\beta}n^{2\beta},\\
     & \leq C_{\beta}n^{2\beta}.\\
  \end{aligned}
  \end{equation}
假定存在一个常数,独立于 $T$,对于任意 $s,t \geq 0$,有
\begin{equation}
\begin{aligned}
\mathbb{E}[|H_{t}-H_{s}|^{2}]&=\mathbb{E}[|\int_{0}^{t}\varphi_{i}(t)dG_{t}-\int_{0}^{s}\varphi_{j}(t)dG_{s}|^{2}] \\
& \leq \mathbb{E}[\int_{s}^{t}\int_{s}^{t}dG_{t}dG_{s}]\\
& \leq \int_{s}^{t}\int_{s}^{t}|x-y|^{2\beta-2}dxdy +\int_{s}^{t}\int_{s}^{t}|xy|^{\beta-1}dxdy \\
& \leq C_{\beta}^{'}|t-s|^{2\beta}
\end{aligned}
\end{equation}
当$\beta \in (\frac{1}{2},1)$,
由切比雪夫不等式和多重 Wiener-It$\hat{o}$ 积分的超收缩性可以推出,对于任意 $\varepsilon >0$ 和 $p(1-\beta )>-1$,
\begin{equation}
p(\frac{H_{n}}{n}>\varepsilon )\leq \frac{EH_{n}^{p}}{n^{ p}\varepsilon ^{p}}%
\leq \frac{C(EH_{n}^{2})^{p/2}}{n^{p}\varepsilon ^{p}}\leq \frac{C}{%
n^{p(1-\beta)}}
\end{equation}
由 Borel-Cantelli 引理可以推出$\beta \in (\frac{1}{2},1)$,
\begin{equation}
 \lim_{n\rightarrow \infty }\frac{H_{m}}{m}=0,a.s.
\end{equation}
其次,存在两个独立于 $n$ 的常数 $\alpha \in (0,1)$ 和 $C_{\alpha ,\beta }>0$ ,那么对于任意的 $|t-s|\leq 1$,
\begin{equation}
E[|H_{t}-H_{s}|^{2}]\leq C_{\alpha ,\beta }|t-s|^{2\beta}
\end{equation}
然后, Garsia-Rumsey 不等式意味着对于任意实数 $p>\frac{4}{\alpha }$, $q>1$, 和整数
$n\geq 1$,
\begin{equation}
|H_{t}-H_{s}|\leq R_{p,q}m^{q/p},  \forall t,s\in[m,m+1]
\end{equation}
其中 $R_{p,q}$ 是独立于 $m$ 的随机常数。
最后,由于 $|\frac{H_{n}}{n}|\leq \frac{1}{n}|H_{n}-H_{m}|+\frac{m}{n}\frac{|H_{m}|}{m}$.
其中 $m=[n]$ 是小于或等于实数 $n$ 的最大整数,随着$n\rightarrow \infty $,可以得到 $\frac{H_{n}}{n}$ 几乎处处收敛到0。

则$\frac{1}{n}R_{n}$中第 $p+1$ 维分量的为
\begin{equation}
\begin{aligned}
  \lim_{n \to \infty}\frac{1}{n}\int_{0}^{n}X_{t}dG_{t}&= \lim_{n \to \infty}\frac{1}{n}\int_{0}^{n}(h(t) + Z_{t}) dG_{t}\\
  &=\lim_{n \to \infty}\frac{1}{n}\int_{0}^{n}h(t)dG_{t}+\lim_{n \to \infty}\frac{1}{n}\int_{0}^{n}Z_{t}dG_{t} \\
\end{aligned}
\end{equation}
其中第一项因$h(t)$是带周期的有界函数,可由前部分$H_{n}$可知几乎处处收敛到0.
记
\begin{equation}
E_{1}=\frac{1}{n}\int_{0}^{n}Z_{t}dG_{t}=\frac{1}{n}\int_{0}^{n}\int_{0}^{n}e^{-k(t-s)}dG_{s}dG_{t}
\end{equation}
由\cite{Yong.C 2020}中proposition 3.8中,
\begin{equation*}
\lim_{T \to \infty}\frac{F_{T}}{T}=0, a.s.
\end{equation*}
其中,
\begin{equation}
F_{t}:=I_{2}(f_{T}(r,s)\mathds{1}_{0\leq r,s \leq t})=\int_{0}^{s}\int_{0}^{t}e^{-k(t-s)}dG_{s}dG_{t}
\end{equation}
当$n \to \infty$,可得到 $E_{1}$ 几乎处处收敛到0,即$\frac{1}{n}R_{n}$几乎处处收敛到0。
\end{proof}
\end{proposition}
\begin{proposition}\label{423}
  当 $n\rightarrow\infty$,可得 $nQ_{n}^{-1}$ 几乎处处收敛到
  $$
  C:=
  \begin{pmatrix}
    I_p + \gamma \Lambda \Lambda^t & \gamma \Lambda \\
    \gamma \Lambda^t                 &  \gamma          \\
  \end{pmatrix}
  $$
这里
\begin{equation}
\begin{aligned}
  \Lambda=(\Lambda_1,...,\Lambda_p)^t:&=(\int_{0}^{1}\varphi_{1}(t)h(t)dt,...,\int_{0}^{1}\varphi_{p}(t)h(t)dt)^t\\
  \gamma &=(\int_{0}^{t}h(t)^{2}dt+k^{-2\beta}\Gamma{2\beta-1}-\sum_{i=1}^{p}\Lambda_{i}^{2})^{-1}\\
\end{aligned}
\end{equation}
\begin{proof}
  这里分别讨论矩阵 $n^{-1}Q_{n}$ 里各元素的极限性质
首先考虑$\Lambda_{n,i}$
\begin{equation}
\begin{aligned}
  \lim_{n \to \infty} \Lambda_{n,i}&=\lim_{n \to \infty} \frac{1}{n}\int_{0}^{n}X_{t}\varphi_{i}(t)dt \\
  &=\lim_{n \to \infty} \frac{1}{n}\int_{0}^{n}h(t)\varphi_{i}(t)dt + \lim_{n \to \infty} \frac{1}{n}\int_{0}^{n}\varphi_{i}(t)\cdot Z_{t}dt\\
  &=\int_{0}^{1}h(t)\varphi_{i}(t)dt+ \lim_{n \to \infty} \frac{1}{n}\int_{0}^{n}\int_{0}^{t}\varphi_{i}(t)e^{-k(t-s)}dG_{s}dt
\end{aligned}
\end{equation}
最后一个等式中的第二项 ,
由\cite{Pei 2020}中的 proposition 3.6 可知
\begin{equation}
\lim_{T \to \infty} \frac{F_{T}}{T^\gamma}=0,a.s.
\end{equation}
其中,有$\gamma > \beta,\beta\in(\frac{1}{2},1)$ 和
\begin{equation}\label{4210}
F_{T}=\int_{0\leq s\leq t }e^{-k(t-s)}dG_{s}dt
\end{equation}
可得$\lim_{n \to \infty} \frac{1}{n}\int_{0}^{n}\int_{0}^{t}\varphi_{i}(t)e^{-k(t-s)}dG_{s}dt$ 为0。

然后考虑
\begin{equation}\label{427}
\begin{aligned}
  \lim_{n \to \infty} \frac{1}{n}\int_{0}^{n}X_{t}^{2}dt&=\lim_{n \to \infty} \frac{1}{n}\int_{0}^{n}h(t)^{2}dt +\lim_{n \to \infty}\frac{1}{n}\int_{0}^{n}Z(t)^2dt + \lim_{n \to \infty}\frac{2}{n}\int_{0}^{n}h(t)Z_{s}dt\\
  &=\int_{0}^{1}h(t)^{2}dt +\sigma^{2}k^{-2\beta}\Gamma(2\beta-1)\\
\end{aligned}
\end{equation}
由上式\ref{4210},且$h(t)$ 为有界函数,故这里交叉项为0。

因此 $\gamma_{n}$ 可以表示为
\begin{equation}
\begin{aligned}
  \lim_{n \to \infty}\gamma_{n}&=\lim_{n \to \infty}(\frac{1}{n}\int_{0}^{n}X_{t}^{2}dt-\sum_{i=1}^{p}\Lambda_{n,i}^{2})^{-1}\\
  &=(\int_{0}^{1}h(t)^{2}dt +\sigma^{2}k^{-2\beta}\Gamma(2\beta-1)-\sum_{i=1}^{p}\Lambda_{i}^{2})^{-1}
\end{aligned}
\end{equation}
由于函数 $\varphi_{i};i=1,...,p$ 是在 $L^{2}[0,1]$的正交基,用Bessel不等式可以求得
\begin{equation}
\begin{aligned}
  \sum_{i=1}^{p}\Lambda_{i}^{2}=\sum_{i=1}^{p}(\int_{0}^{1}\varphi_{i}(t)h(t)dt)^{2} \leq \int_{0}^{1}h(t)^{2}dt
\end{aligned}
\end{equation}
即为所证
\end{proof}
\end{proposition}
\section{最小二乘估计量的渐近正态性}
\label{section:4.3}
根据$G$原先的假定来处理 $\theta$ 的渐近性质,其核心步骤为三步:

第一步为:
\begin{equation*}
\mathbb{E}[n^{-\beta}\int_{0}^{n}Z_{t}dG_{t}]=0;
\end{equation*}

第二步为:
\begin{equation*}
\mathbb{E}[\int_{0}^{n}\varphi_{i}(t)dG_{t}\int_{0}^{n}Z_{t}dG_{t}]=0;
\end{equation*}

第三步为:
\begin{equation*}
\mathbb{E}[\int_{0}^{n}f_{k}(s)dG_{t}\int_{0}^{n}f_{k}(t)dG_{t}].
\end{equation*}
收敛到正态分布,在这里有可能是原先 $G_{t}$ 的假定中,将$(t+s)^{2} \leq 2ts$,放太大,使得$\int_{0}^{n}\int_{0}^{n}f_{k}(x)f_{l}(y)|xy|^{\beta-1}dxdy$不收敛。故无法得出具体的渐近分布。

这里尝试用次分数布朗运动、双布朗运动、广义次分数布朗运动特殊的例子来考虑其渐近性质,并对高斯过程协方差函数的二阶偏导形式作出重新如下假定,并使得这个三个特殊例子均满足如下假定,故统一处理。
\begin{Assumption}\label{assumption:4.3.1}
对于$\beta \in (\frac{1}{2},1)$, 任意 $t\neq s\in \lbrack 0,\infty )$ 的协方差函数为 $R(t,s)=E[G_{t}G_{s}]$ 。
\begin{equation*}
    \frac{\partial ^{2}}{\partial t\partial s}R(t,s) \leq C_{\beta }|t-s|^{2\beta-2}
\end{equation*}
其中常数 $\beta ,C_{\beta }>0$ 不依赖于T。此外,对于任意 $t\geq 0$, 有$R(0,t)=0$。
\end{Assumption}

\begin{example}
次分数布朗运动 ${S^{H}(t),t \ge 0}$,其参数为$H\in(0,1)$,其协方差函数为:
\begin{equation}
 R(t,s)=s^{2H}+t^{2H}-\frac{1}{2}((s+t)^{2H}+|t-s|^{2H})
\end{equation}
当$\beta=H>\frac{1}{2}$,根据\cite{Mendy I 2013}中的Lemma 2.1可得,满足assumption1.1。
\end{example}

\begin{example}
双分数布朗运动${B^{H,K}(t),t \ge 0}$,其指数$H,K\in(0,1)$,其协方差函数为:
\begin{equation}
R(t,s)=\frac{1}{2^{k}}((s^{2H}+t^{2H})^{k}-|t-s|^{2HK})
\end{equation}
当$\beta=HK >\frac{1}{2}$,满足assumptin1.1。
\end{example}
\begin{example}
广义次分数布朗运动 $S^{H,K}(t)$, $H\in(0,1),k\in[1,2),HK\in(0,1)$,其协方差函数为:
\begin{equation}
R(t,s)=(s^{2H}+t^{2H})^{K}-\frac{1}{2}[(t+s)^{2HK}+|t-s|^{2HK}]
\end{equation}
当$\beta:HK>\frac{1}{2}$,满足assumption1.1.[请参考\cite{Sghir A 2013}]
\end{example}

 这里需要如下两个相关命题相关结论来证明 $\theta$ 的渐近性质
\begin{proposition}\label{proposition435}
令 $F_{T}=\int_{0}^{n}\int_{0}^{n}e^{-\theta(t-s)}dG_{s}dG_{t}$,$f_{T}=e^{-\theta|t-s|}$,
当$\beta\in(\frac{1}{2},\frac{3}{4})$ 时,可知
\begin{equation}
\lim_{T \to \infty}\frac{1}{4\theta\sigma_{\beta}^{2}T}\mathbb{E}(F_{T}^{2})=(C_{\beta}\Gamma(2\beta-1)\theta^{-2\beta})^2.
\end{equation}
\begin{proof}
由 $It\hat{o}$ 的等距性,
\begin{equation}
\mathbb{E}(|F_{T}|^{2})=2\left\Vert f_{T}\right\Vert_{\mathfrak{H}^{\bigotimes{2}}}^2
\end{equation}
由式(3.2.8)可得,
\begin{equation}\label{equation422}
|\left\Vert\phi\right\Vert_{\mathfrak{H}^{\bigotimes{2}}}^{2}-\left\Vert\phi\right\Vert_{\mathfrak{H}_{1}^{\bigotimes{2}}}^{2}|\leq\left\Vert\phi\right\Vert_{\mathfrak{H}_{2}^{\bigotimes{2}}}^{2}+2C_{\beta}^{'}\left\Vert K\varphi\right\Vert_{\mathfrak{H}_{1}}^{2}
\end{equation}
由 \cite{Yong.C 2020}中proposition3.4可得,
\begin{equation}
\begin{aligned}
\lim_{T \to \infty}\frac{1}{2\theta \sigma_{\beta}^{2}T}\left\Vert f_{T}\right\Vert_{\mathfrak{H}_{1}^{\bigotimes{2}}}=(C_{\beta}\Gamma(2\beta-1)\theta^{-2\beta})^2,\\
\left\Vert Kf_{T}\right\Vert_{\mathfrak{H}_{1}}^{2} \leq C\int_{0<v<u<T}(u-v)^{2\beta-2}(uv)^{\beta-1}dudv=CT^{4\beta-2}
\end{aligned}
\end{equation}
结合\ref{equation422},可得,当$\beta \in(\frac{1}{2},\frac{3}{4})$,
\begin{equation}
\lim_{T \to \infty}\frac{1}{2\theta \sigma_{\beta}^{2}T}|\left\Vert f_{T}\right\Vert_{\mathfrak{H}^{\bigotimes{2}}}^{2}-\left\Vert f_{T}\right\Vert_{\mathfrak{H}_{1}^{\bigotimes{2}}}^{2}|=0
\end{equation}
因此得出以上结论。
\end{proof}
\end{proposition}
\begin{proposition}\label{proposition436}
令$f_{k},k=1,...,q$是周期为1的函数,对于$1\leq k,l \leq q,i\ge1$,可得
\begin{equation*}
(n^{-\beta}\int_{0}^{n}f_{k}(t)dG_{t})_{1\leq k\leq q}\to N(0,\int_{0}^{1}\int_{0}^{1}f_{k}(x)f_{l}(y)dxdy_{1\leq k,l\leq q}+C).
\end{equation*}
\begin{proof}
左边的式子为高斯向量,因此其收敛到其协方差矩阵。
根据 $G_{t}$ 的假定,
\begin{equation}
\mathbb{E}[\int_{i-1}^{i}f_{k}(t)dG_{t}\int_{i-1}^{i}f_{l}(s)dG_{s}]= \beta(2\beta-1)\int_{0}^{1}\int_{0}^{1}f_{k}(t)f_{l}(s)|t-s|^{2\beta-2}dsdt+C.
\end{equation}
因此,对于任意 $1\leq k,l\leq q$,
\begin{equation}
\begin{aligned}
\mathbb{E}[\int_{0}^{n}f_{k}(s)dG_{s}\int_{0}^{n}f_{l}(t)dG_{t}]&=\sum_{i,j=1}^{n}\mathbb{E}(\int_{i-1}^{i}f_{k}(s)dG_{s}\int_{j-1}^{j}f_{l}(t)dG_{t}) \\
&=\beta(2\beta-1)[n\int_{0}^{1}\int_{0}^{1}f_{k}(x)f_{l}(y)|t-s|^{2\beta-2}dxdy \\
&+\sum_{i,j\neq 1}\int_{0}^{1}\int_{0}^{1}f_{k}(x)f_{l}(y)|j-i+y-x|^{2\beta-2}dxdy]+C\\
\end{aligned}
\end{equation}
由\cite{Salwa Bajja 2017}可知,对于$x,y\in[0,1]$,当$j-i \to \infty$,有$|j-i+y-x|^{2\beta-2} \sim n|j-i|^{2\beta-2}-|j-i|^{2\beta-1}$,和
\begin{equation}
\sum_{i\neq j=1}^{n}n|j-i|^{2\beta-2}-|j-i|^{2\beta-1}=\frac{n^{-2\beta}}{\beta(2\beta-1)}
\end{equation}
又因为当 $j-i\geq 1$,$|j-i+y-x|^{2\beta-2} \leq (j-i-1)^{2\beta-2}$,且$\sum_{i=1}^{n-1}\sum_{j=i+1}^{n}(j-i-1)^{2\beta-2} < \infty$.则由控制收敛定理可得
\begin{equation}
\sum_{i\neq j=1}^{n}|j-i+y-x|^{2\beta-2} \to \sum_{i\neq j=1}n|j-i|^{2\beta-2}-|j-i|^{2\beta-1}
\end{equation}
因此,得出结论,
\begin{equation}
n^{-2\beta}\mathbb{E}[\int_{0}^{n}f_{k}(s)dG_{s}\int_{0}^{n}f_{l}(t)dG_{t}]\to \int_{0}^{1}\int_{0}^{1}f_{k}(x)f_{l}(y)dxdy+C.
\end{equation}
\end{proof}
\end{proposition}
\begin{theorem}
假定$\beta\in(\frac{1}{2},\frac{3}{4})$,可以得出
\begin{equation}\label{431}
n^{1-\beta}(\hat{\theta_{n}}-\theta)\to N(0,C^{T} \Sigma C)
\end{equation}
其中,矩阵 $C$ 如\ref{423}所定义,
$$
  \Sigma:=
  \begin{pmatrix}
    G & -a \\
    -a^{\top} & b\\
  \end{pmatrix}
  $$
这里
\begin{equation*}
\begin{aligned}
G:&=(\int_{0}^{1}\int_{0}^{1}\varphi_{i}(u)\varphi_{j}(v)dudv)_{1\leq i,j\leq p}+C;\\
a^{\top}:&=(\beta(2\beta-1)\int_{0}^{1}\int_{0}^{1}\varphi_{i}(u)h(v)|v-u|^{2\beta-2}dudv)_{1\leq i \leq p};\\
b:&=\beta(2\beta-1)\int_{0}^{1}\int_{0}^{1}h(u)h(v)|v-u|^{2\beta-2}dudv.
\end{aligned}
\end{equation*}
\begin{proof}
由(4.0.12)可得,$n^{1-\beta}(\hat{\theta_{n}}-\theta)=(nQ_{n}^{-1})\cdot(n^{-\beta}R_n)$,

由\ref{423} 可知,$nQ_{n}^{-1}$ 几乎处处收敛到 $C$,要证明\ref{431}成立,即需证明,当$n \to \infty$ 时,
\begin{equation*}
n^{-\beta}R_{n}=(n^{-\beta}\int_{0}^{n}\varphi_{1}(t)dG_{t},...,n^{-\beta}\int_{0}^{n}\varphi_{p}(t)dG_{t},-n^{-\beta}\int_{0}^{n}X_{t}dG_{t})\to N(0,\Sigma).
\end{equation*}

由\cite{Peccati 2005}和\cite{Peccati 2007}的主要结论,即对于一个一般高斯过程驱动下的多重维纳伊藤积分的K维向量$I(l)$,这k维高斯向量的渐近分布与其协方差矩阵相同。由于$n^{-\beta}R_{n}$ 是一个多重积分分量组成的向量,则随着 $n\to \infty$,$n^{-\beta}R_{n}$收敛到其协方差阵。

step1:将\ref{40.2}中的$X_{t}$ 带入向量中的最后一个分量,则有
\begin{equation}
n^{-\beta}\int_{0}^{n}X_{t}dG_{t}=n^{-\beta}\int_{0}^{n}Z_{t}dG_{t}+n^{-\beta}\int_{0}^{n}h(t)dG_{t}
\end{equation}
其中由命题 \ref{proposition435},且$-2\beta \in (-\frac{3}{2},-1)$可得
\begin{equation}
\mathbb{E}[(n^{-\beta}\int_{0}^{n}Z_{t}dG_{t})^2]=0,
\end{equation}

因为为三重的维纳积分,由\cite{Nourdin P 2012}中的多重维纳积分的等距性可知,
\begin{equation}
\mathbb{E}[n^{-2\beta}(\int_{0}^{n}Z_{t}dG_{t})(\int_{0}^{n}\varphi_{i}(t)dG_{t})]=0.
\end{equation}
即在 $X_{t}=h(t)+Z_{t}$,其中 $Z_{t}$在协方差阵中的相关项为0.

step2:那么,需证明
\begin{equation}
(n^{-\beta}\int_{0}^{n}\varphi_{1}(t)dG_{t},...,n^{-\beta}\int_{0}^{n}\varphi_{p}(t)dG_{t},-n^{-\beta}\int_{0}^{n}h_{t}dG_{t})\to N(0,\Sigma)
\end{equation}
其中$h(t),\varphi_{i},i=1,...,p$是周期为1的有界函数,由\ref{proposition436}得证。
\end{proof}
\end{theorem}
\section{小结}
在\ref{section:4.3}中第三步为:
\begin{equation*}
\mathbb{E}[\int_{0}^{n}f_{k}(s)dG_{t}\int_{0}^{n}f_{k}(t)dG_{t}].
\end{equation*}
收敛到正态分布。

前两个步骤已求证,在第三步中,根据 $G_{t}$的假定,其协方差函数的二阶偏导形式由两部分 $|t-s|^{2\beta-2}$ 和$|ts|^{\beta-1}$ 组成,在求解过程中
\begin{equation}
n^{-2\beta} \int_{0}^{n}\int_{0}^{n}f_{k}(t)f_{l}(s                                  )|t-s|^{2\beta-2}dtds\to \int_{0}^{1}\int_{0}^{1}f_{k}(x)f_{l}(y)dxdy
 \end{equation}
 而此式 $n^{-2\beta}\int_{0}^{n}\int_{0}^{n}f_{k}(x)f_{l}(y)|xy|^{\beta-1}dxdy$ 不收敛,故无法得出具体的渐近分布。
\subsection{解决方案}
可能由于 $(t+s)^{2} \leq 2t^{2}s^{2}$ 放太大使其无法收敛,故直接考虑
 \begin{equation}
 \int_{0}^{n}\int_{0}^{n}f_{k}(x)f_{l}(y)(x+y)^{2\beta-2}dxdy
 \end{equation}
 即需要证明 $n^{-2\beta}\int_{0}^{n}\int_{0}^{n}f_{k}(x)f_{l}(y)(x+y)^{2\beta-2}dxdy $ 的收敛性
\begin{equation}
\begin{aligned}
\int_{0}^{n}\int_{0}^{n}(x+y)^{2\beta-2}dxdy
&=\int_{0}^{n}\frac{1}{2\beta-1}[(y+n)^{2\beta-1}-s^{2\beta-1}]dy\\
&=\frac{4^{\beta}-2}{(2\beta-1)2\beta}\cdot n^{2\beta}
\end{aligned}
\end{equation}
记$C_{\beta}:\frac{4^{\beta}-2}{(2\beta-1)2\beta}$,
故
\begin{equation}
n^{-2\beta}\int_{0}^{n}\int_{0}^{n}(x+y)^{2\beta-2}dxdy \to C_{\beta}
\end{equation}
可知,$\int_{0}^{n}\int_{0}^{n}f_{k}(x)f_{l}(y)(x+y)^{2\beta-2}dxdy$不会收敛到0。

因为 $f_{k}(x),f_{l}(y)$是周期为1的实值有界函数,设$|f_{k}(x)|\leq M$,$|f_{l}(y)|\leq N$,由单调有界定理可得
\begin{equation}
\begin{aligned}
n^{-2\beta}\int_{0}^{n}\int_{0}^{n}f_{k}(x)f_{l}(y)(x+y)^{2\beta-2}dxdy
&\leq M \cdot N \cdot n^{-2\beta}\int_{0}^{n}\int_{0}^{n}(x+y)^{2\beta-2}dxdy
&=M \cdot N\cdot C_{\beta}
\end{aligned}
\end{equation}
记$C:=M \cdot N\cdot C_{\beta}$因此,当$n\to \infty$时,有
\begin{equation}
n^{-2\beta}\int_{0}^{n}f_{k}(s)dG_{t}\int_{0}^{n}f_{k}(t)dG_{t}\to N(0,\int_{0}^{1}\int_{0}^{1}f_{k}(x)f_{l}(y)dxdy+C)
\end{equation}
\chapter{总结与展望}

  在这篇文章,首先研究了在一般高斯过程驱动下的Vasicek模型,其参数在两种形式下的强相合性和渐近正态性,以及在一般高斯过程驱动下的带周期均值的O-U过程,其参数估计量的相合性和渐近性质。

  首先,我们对一般高斯过程协方差函数的二阶偏导形式做出假定,并使得分数布朗运动以及其他一些高斯过程满足此假定。主要考虑当 $k>0$ 和 $\beta>\frac{1}{2}$ 时,基于不断增加的时间跨度即 ($[0,T]$,
  $T\to\infty$)的连续时间上观测值 $X_{t}$,我们给出模型的解以及参数的矩估计量。接下来根据最小二乘思想,得到 $k$ 和 $\mu$ 的最小二乘估计量。我们根据希伯尔空间上的内积表达式和四阶矩定理,得到了如下结果:四个估计量的强相合性和渐近正态性。在未来,在这些结论的基础上,我们将尝试给出这四个估计量的收敛速度。

  然后,对本文的第四章,我们研究了在一般高斯过程驱动下的带周期均值的O-U过程。确切的说,假设一般高斯过程协方差函数的二阶偏导形式做出假定,漂移系数项变成带周期均值的函数$L(t)=\sum_{i=1}^{p}\mu_{i}\varphi_{i}(t)$,我们使用最小二乘方法估计参数 $\theta$,考虑随机微分方程的 $p+1$ 维的参数向量的估计问题,获得的结果表明:对于任意 $\beta \in(\frac{1}{2})$,$\theta$的最小二乘估计量是强相合的,然而其渐近性质未收敛到具体的正态分布。在未来,我们将考虑研究出更确切的渐近正态性性质才足以研究其收敛速度。


%%------------------------------------------参考文献------------------------------------------ %%

\begin {thebibliography}{99} \small

% \small \songti\zihao{-4}
\bibitem{Yong.C 2020}
 Chen Y, Zhou H. Parameter estimation for an Ornstein-Uhlenbeck Process driven by a general Gaussian noise[J]. arXiv preprint arXiv:2002.09641, 2020.

\bibitem{Vasicek.O 1977}
 Vasicek O. An equilibrium characterization of the term structure[J]. Journal of financial economics, 1977, 5(2): 177-188.
\bibitem{Hu.Y and Nualart 2010}
 Hu Y, Nualart D. Parameter estimation for fractional Ornstein–Uhlenbeck processes[J]. Statistics probability letters, 2010, 80(11-12): 1030-1038.

\bibitem{Jingzhi.H 2012}
 Huang J Z, Huang M. How much of the corporate-treasury yield spread is due to credit risk?[J]. The Review of Asset Pricing Studies, 2012, 2(2): 153-202.

\bibitem{San. W 2020}
 Wu S, Dong Y, Lv W, et al. Optimal asset allocation for participating contracts with mortality risk under minimum guarantee[J]. Communications in Statistics-Theory and Methods, 2020, 49(14): 3481-3497.

\bibitem{Yong.C 2017}
 Chen Y, Hu Y, Wang Z. Parameter estimation of complex fractional Ornstein-Uhlenbeck processes with fractional noise[J]. arXiv preprint arXiv:1701.07568, 2017.
\bibitem{Jolis M 2007}
 Jolis M. On the Wiener integral with respect to the fractional Brownian motion on an interval[J]. Journal of mathematical analysis and applications, 2007, 330(2): 1115-1127.

\bibitem{Nourdin D(2006)}
 Nualart D. The Malliavin calculus and related topics[M]. Berlin: Springer, 2006.

\bibitem{Nualart D(2005)}
 Nualart D, Peccati G. Central limit theorems for sequences of multiple stochastic integrals[J]. The Annals of Probability, 2005, 33(1): 177-193.

\bibitem{yang B 2013}
 Yang B H. Estimating Long-Run PD, Asset Correlation, and Portfolio Level PD by Vasicek Models[J]. 2013.

\bibitem{Fergusson K 2015}
 Fergusson K, Platen E. Application of maximum likelihood estimation to stochastic short rate models[J]. Annals of Financial Economics, 2015, 10(02): 1550009.

\bibitem{Xiao Yu 2017}
 Xiao W, Yu J. Asymptotic theory for estimating drift parameters in the fractional Vasicek model[J]. 2017.
\bibitem{Xiao Zhang Zou 2018}
 Xiao W, Zhang X, Zuo Y. Least squares estimation for the drift parameters in the sub-fractional Vasicek processes[J]. Journal of Statistical Planning and Inference, 2018, 197: 141-155.
\bibitem{Nourdin Tran 2018}
 Nourdin I, Tran T T D. Statistical inference for Vasicek-type model driven by Hermite processes[J]. Stochastic Processes and their Applications, 2019, 129(10): 3774-3791.
\bibitem{Yu 2018}
 Yu Q. Statistical inference for Vasicek-type model driven by self-similar Gaussian processes[J]. Communications in Statistics-Theory and Methods, 2020, 49(2): 471-484.
\bibitem{peng2008}
 彭秀丹. 我国利率模型的参数估计与选择[J]. 2008.
\bibitem{Geman2005}
 Helyette Geman. Commodities and Commodity Derivatives,Wiley Finance[J].  2007.
\bibitem{Franke B 2013}
 Franke B,Kott T.Parameter estimation for the drift of a time inhomogeneous jump diffusion process[J]. Statistica Neerlandica, 2013,67(2):145-168.
\bibitem{Salwa Bajja 2017}
 Bajja S , Es-Sebaiy K , Viitasaari L . Least squares estimator of fractional Ornstein Uhlenbeck processes with periodic mean[J]. Journal of the Korean Statistical Society, 2016.
\bibitem{Dehling 2016}
 Dehling H , Franke B , Woerner J H C . Estimating drift parameters in a fractional Ornstein Uhlenbeck process with periodic mean[J]. Statistical Inference for Stochastic Processes, 2015, 20(1):1-14.
\bibitem{Mendy I 2013}
 Mendy I . Parametric estimation for sub-fractional Ornstein–Uhlenbeck process[J]. Journal of Statistical Planning Inference, 2013, 143(4).
\bibitem{Sghir A 2013}
 Aissa, Sghir. The generalized sub-fractional Brownian motion[J]. Communications on Stochastic Analysis, 2013.
\bibitem{Peccati 2005}
 Peccati G , Tudor C A . Gaussian limits for vector-valued multiple stochastic integrals[J]. Séminaire De Probabilités XXXVIII, 2005, 1857:219-245.
\bibitem{Peccati 2007}
 Peccati G ,Gaussian Approximations of Multiple Integrals.Electronic Communications in Probability,12,350-364.
\bibitem{Pei 2020}
 Xingzhi Pei,Parameter estimation for Vasicek model driven by a general Gaussian noise.
\bibitem{Nourdin P 2012}
 Nourdin I,Peccati G, Normal Approximations with Malliavin Calculus:from Stein's method to university[M].Cambridge University Press,Cambridge.
\bibitem{Chen 2017}
 Y.Chen,Y Hu, and Z.Wang,Parameter estimation of complex fractional Ornstein-Uhlenbeck processes
 with fractional noise.2017
\end {thebibliography}

%%-------------------------------------致谢-------------------------------%%
\chapter*{致 \ \ 谢}
% \addcontentsline{toc}{chapter}{致 \ \ 谢}

本论文是在导师陈勇老师的悉心指导下完成的。首先我要衷心的感谢导师,是他将我引入这个有趣且具有挑战性的研究领域。导师渊博的专业知识,严谨无私的治学精神以及精益求精的人生态度都潜移默化的对我产生了深远的影响。在硕三年的成长,将是我一生宝贵的财富;在硕三年导师的关怀,我永生难忘。

同时也要感谢江西师范大学数学与统计学院为我们提供了良好的学习环境以及对外学习交流的机会。

幸有师大相伴,幸遇恩师指点。









\

\

\

 \

  \qquad\qquad\qquad\qquad\qquad\qquad\qquad   \qquad\qquad\qquad \qquad\qquad  裴\  幸\ 智


\qquad\qquad\qquad\qquad\qquad\qquad\qquad    \qquad\qquad\ \ \qquad
\qquad \qquad   2021.3

% 于江西师范大学
%%-------------------------------------硕士期间研究成果-------------------------------%%
\chapter*{硕士期间公开发表论文及科研情况}
% \addcontentsline{toc}{chapter}{硕士期间公开发表论文及科研情况}
[1] Parameter estimation for Vasicek model driven by a general Gaussian noise.[在投]

% 于江西师范大学
%%-------------------------------------硕士期间研究成果-------------------------------%%
\end{document}
