\documentclass[reqno,a4paper,10pt]{amsart}
% \usepackage{amsmath}
% \usepackage{amsthm}
\usepackage{ctex}
\usepackage{mathtools}
\newcommand\me{\mathrm{e}}
\newcommand\mi{\mathrm{i}}
\usepackage{mathtools}
%\usepackage[greek,english]{babel}
\usepackage{amssymb}
\usepackage{zhlipsum}
\newcommand\dif{\,\mathrm{d}}
\newcommand\mE{\mathbb{E}}
\newcommand\Pro{\mathbb{P}}
\newcommand\Rnum{\mathbb{R}}
\newcommand{\Kov}{柯尔莫戈洛夫}
\makeatletter
\@addtoreset{equation}{section}
\makeatother
\renewcommand{\theequation}{\arabic{section}.\arabic{equation}}
\newcommand{\abs}[1]{\left\vert#1\right\vert}
\usepackage{hyperref}
\hypersetup{hypertex=true,
            colorlinks=true,
            linkcolor=cyan,
            anchorcolor=black,
            citecolor=black,
            pdfauthor={丁珍,程预敏},% 作者
            pdftitle={测度论练习题3},% 标题
            pdfsubject={概率论},% 主题
            pdfkeywords={测度论,Lebesgue测度,Lebesgue积分},% 关键字
            pdfproducer={LaTeX},
            pdfcreator={xeLaTeX}}

\usepackage{mathrsfs}
%\usepackage{ntheorem}
%\theoremsymbol{\mbox{$\box$}}
\newtheorem{theorem}{定理}[section]
\newtheorem{lemma}[theorem]{引理}
\newtheorem{proposition}[theorem]{命题}
\newtheorem{corollary}[theorem]{推论}
\newtheorem{definition}[theorem]{定义}
\newtheorem{example}{例}
\newtheorem{remark}[theorem]{注}
\newtheorem*{prove}{\textbf{解}}
\newenvironment{solution}{\begin{proof}[\indent\bf 解]}{\end{proof}}
\renewcommand{\proofname}{\indent\bf 证明}
\newtheorem*{thm2}{Proof of Theorem 2.2}
\usepackage{geometry}
\geometry{a4paper,left=2cm,right=2cm,top=2.5cm,bottom=2.5cm}


\title{\textbf{测度论导论\S1.3习题}}
\author{丁\;\;\; 珍}
    % \address{201940100486}
\author{程预敏}
    % \address{201940100447}

\begin{document}
\maketitle


\section{Solution of Ex 1.3.2}
(Basic properties of the complex-valued simple integral). Let $f,g: \mathbb{R}^d\to \mathbb{C}$ be absolutely integrable simple functions.

(i) (*-linearity) We have 
\begin{equation}
    \mathrm{Simp}\int_{\mathbb{R}^d} f(x)+g(x)\dif x= \mathrm{Simp}\int_{\mathbb{R}^d} f(x)\dif x+\mathrm{Simp}\int_{\mathbb{R}^d} g(x)\dif x
    \label{additivity}
\end{equation}
and 
\begin{equation}
    \mathrm{Simp}\int_{\mathbb{R}^d} c f(x)\dif x=c\times\mathrm{Simp} \int_{\mathbb{R}^d} f(x)\dif x
    \label{dot_multi}
\end{equation}
for all $c\in \mathbb{C}$. Also we have 
\begin{equation}
    \mathrm{Simp} \int_{\mathbb{R}^d} \overline{f}(x)\dif x=\overline{\mathrm{Simp}\int_{\mathbb{R}^d} f(x)\dif x}.
    \label{conjugation}
\end{equation}
\begin{proof}
    我们首先考虑绝对收敛的实数值简单函数的情形,即$f,g: \mathbb{R}^d\to \mathbb{R}$。由于$f,g$都是简单函数,则$h=f+g$也是简单函数。且
    \begin{equation*}
        h_+ - h_-=(f_+ - f_-)+(g_+ - g_-).
    \end{equation*}
    由于$f,g$都是绝对收敛的简单函数,则易证上面的函数是绝对收敛的非负简单函数。且我们有
    \begin{equation}
        h_+ + f_- + g_- = h_- + f_+ + g_+.
        \label{equivalence}
    \end{equation}
    根据非负简单函数积分的线性性,我们有
    \begin{equation*}
        \begin{split}
            \mathrm{Simp}\int_{\mathbb{R}^d} h_+(x)\dif x + \mathrm{Simp}\int_{\mathbb{R}^d} f_-(x)\dif x +\mathrm{Simp}\int_{\mathbb{R}^d} g_-(x)\dif x &=\\
            \mathrm{Simp}\int_{\mathbb{R}^d} h_-(x)\dif x + \mathrm{Simp}\int_{\mathbb{R}^d} f_+(x)\dif x &+ \mathrm{Simp}\int_{\mathbb{R}^d} g_+(x)\dif x.
        \end{split}
    \end{equation*}
    将上式整理后,就可以得到可加性 \eqref{additivity}。
    
    当$c\geq 0\in\mathbb{R}$时,显然有
    \begin{equation*}
        \begin{split}
            \mathrm{Simp}\int_{\mathbb{R}^d} c f(x)\dif x &= \mathrm{Simp}\int_{\mathbb{R}^d} c f_+(x)\dif x -\mathrm{Simp}\int_{\mathbb{R}^d} c f_-(x)\dif x\\
            &=c\times\mathrm{Simp} \int_{\mathbb{R}^d} f_+(x)\dif x - c\times\mathrm{Simp} \int_{\mathbb{R}^d} f_-(x)\dif x\\
            &=c\times\mathrm{Simp} \int_{\mathbb{R}^d} f(x)\dif x.
        \end{split}
    \end{equation*}
    当$c=-1$时,我们有
    \begin{equation*}
        \begin{split}
            \mathrm{Simp}\int_{\mathbb{R}^d} - f(x)\dif x &= \mathrm{Simp}\int_{\mathbb{R}^d}  f_-(x)\dif x -\mathrm{Simp}\int_{\mathbb{R}^d}  f_+(x)\dif x\\
            &=-1\times\mathrm{Simp} \int_{\mathbb{R}^d} f(x)\dif x.
        \end{split}
    \end{equation*}
    根据上面两式,不难证明 \eqref{dot_multi} 对所有$c\in \mathbb{R}$成立。同时,根据复数值简单函数的积分定义,不难将上面两个性质推导到复数值简单函数的情形。

    当$f$是绝对收敛的复数值简单函数,且$f=\sum_{i=1}^n c_i 1_{E_i}$时,易证
    \begin{equation}
        \mathrm{Simp}\int_{\mathbb{R}^d}  f(x)\dif x = \sum_{i=1}^n c_i \cdot m(E_i).
        \label{Uniqueness}
    \end{equation}
    这时我们有$\overline{f}=\overline{\sum_{i=1}^nc_i 1_{E_i}}=\sum_{i=1}^n \overline{c_i}1_{E_i}$,且
    \begin{equation*}
        \mathrm{Simp} \int_{\mathbb{R}^d} \overline{f}(x)\dif x = \sum_{i=1}^n \overline{c_i}\cdot m(E_i)=\overline{\mathrm{Simp}\int_{\mathbb{R}^d} f(x)\dif x}.
    \end{equation*}
\end{proof}
(ii) (Equivalence) If $f$ and $g$ agree almost everywhere, then we have \\$\mathrm{Simp}\int_{\mathbb{R}^d} f(x)\dif x=\mathrm{Simp}\int_{\mathbb{R}^d} g(x)\dif x$.
\begin{proof}
    我们首先考虑实数值简单函数的情形。若$f$和$g$是几乎处处相等的绝对收敛的实数值简单函数,则$f_+$和$g_+$、$f_-$和$g_-$也几乎处处相等。这时,根据非负简单函数的相等性,我们有
    \begin{equation*}
        \begin{split}
            \mathrm{Simp}\int_{\mathbb{R}^d} f_+(x)\dif x = \mathrm{Simp}\int_{\mathbb{R}^d} g_+(x)\dif x,\;\;\;\mathrm{Simp}\int_{\mathbb{R}^d} f_-(x)\dif x = \mathrm{Simp}\int_{\mathbb{R}^d} g_-(x)\dif x.
        \end{split}
    \end{equation*}
    结合上式,我们有$\mathrm{Simp}\int_{\mathbb{R}^d} f(x)\dif x=\mathrm{Simp}\int_{\mathbb{R}^d} g(x)\dif x$。
    
    当$f$和$g$是几乎处处相等的绝对收敛的复数值简单函数时,$Re(f)$和$Re(g)$、$Im(f)$和$Im(g)$也几乎处处相等,根据实数值简单函数的相等性,可以得到$f$和$g$的积分值相等。
\end{proof}
(iii) (Compatibility with Lebesgue measure) For any Lebesgue measurable $E$, one has $\mathrm{Simp} \int_{\mathbb{R}^d} 1_{E}(x)\dif x=m(E)$.
\begin{proof}
    显然,$1_{E}(x)$是非负简单函数,根据非负简单函数的勒贝格测度的相容性,可以得到这一结论。
\end{proof}
(iv) (Uniqueness) Show that the complex-valued simple integral 
\begin{equation*}
    f \to \mathrm{Simp} \int_{\mathbb{R}^d} f(x) \dif x
\end{equation*}
is the only map from the space $\mathrm{Simp}^{abs}(\mathbb{R}^d)$ of absolutely integrable simple functions to $\mathbb{C}$ that obeys all of the above properties.
\begin{proof}
    对于绝对收敛的复数值简单函数$f$,考虑它的典范表示:$f=\sum_{i=1}^n c_i 1_{E_i}$,其中$c_i$各异且$E_i$不交。若$\sum_{j=1}^m s_j 1_{F_j}$也是$f$的一个典范表示。易证在这两个表示中,$m=n$且存在从$i$到$j$的一个双射$\sigma$使得$s_{\sigma(i)}=c_i, F_{\sigma(i)}=E_i$。也就是说,典范表示具有唯一性。根据 \eqref{Uniqueness},我们有
    \begin{equation*}
        \mathrm{Simp}\int_{\mathbb{R}^d}  f(x)\dif x = \sum_{i=1}^n c_i \cdot m(E_i)
    \end{equation*}
    根据典范表示和Lebesgue测度的唯一性,我们可知,绝对收敛的简单函数的简单积分具有唯一性。
\end{proof}



\section{Solution of Ex 1.3.4}\label{1.3.4}
Let $f: \mathbb{R}^d\to [0,+\infty]$. Show that $f$ is a \textit{bounded} unsigned measurable functions if and only if $f$ is the \textit{uniform} limit of \textit{bounded} simple functions.
\begin{proof}
    (i) (充分性) 考虑$\{f_n\}$是一致收敛到$f$的有界非负简单函数列。显然,$f$是非负函数。根据Lemma 1.3.9 (ii),$f$是Lebesgue可测函数。另一方面,对于$\epsilon=1$,存在$n_1\geq 1$使得
    \begin{equation*}
        \sup_{x\in\mathbb{R}^d}\abs{f_{n_1}(x)-f(x)}\leq 1,\;\;\; f(x)\leq f_{n_1}(x)+1,\;\;\; \forall x\in\mathbb{R}^d.
    \end{equation*}
    令$M_{n_1}$是函数$f_{n_1}$的上界,则$f(x)\leq M_{n_1}+1, \forall x\in\mathbb{R}^d$。也就是说$f$是有界非负的可测函数。

    (ii) (必要性) 考虑$f$是有界的非负可测函数,$M\in\mathbb{N}$是它的一个上界。令$\{f_n\}$是如下定义的函数序列:
    \begin{equation*}
        f_n(x)=\sum_{m=0}^{2^n\times M} \frac{m}{2^n}\times 1_{\{\frac{m}{2^n}\leq f(x)<\frac{m+1}{2^n}\}},\;\;\; n\in\mathbb{N}^+.
    \end{equation*}
    显然,由于$f$是可测函数,则$f^{-1}([\frac{m}{2^n},\frac{m+1}{2^n}))$是$\Rnum^d$中的可测集,故$f_n$是有界简单函数,且下列不等式成立:
    \begin{equation*}
        \sup_{x\in\Rnum^d}\abs{f_n(x)-f(x)}\leq \frac{1}{2^n}.
    \end{equation*}
    故,$\{f_n\}$是一致收敛到$f$的有界非负简单函数序列。
\end{proof}

\section{Solution of Ex 1.3.6}\label{1.3.6}
Let $f: \mathbb{R}^d\to [0,+\infty]$ be an unsigned measurable function. Show that the region $\mathcal{D}=\{(x,t)\in \mathbb{R}^d\times \mathbb{R} : 0\leq t\leq f(x)\}$ is a measurable subset of $\mathbb{R}^{d+1}$. 
\begin{proof}
    我们首先考虑简单函数的情况。任给可测集$E$上的示性函数$1_{E}(x)$,显然$\mathcal{D}_{E}=\{(x,t)\in \mathbb{R}^d\times \mathbb{R} : 0\leq t\leq 1_E(x)\}$是由$E\times [0,1]$所生成的可测集。由此,任给简单函数$g(x)$,$\mathcal{D}_{g}=\{(x,t)\in \mathbb{R}^d\times \mathbb{R} : 0\leq t\leq g(x)\}$是可测集。
    
    对于任意非负可测函数$f$,我们记$f^n(x)=\min (f(x),n)$。令$\{f_n\}$是如下定义的函数序列:
    \begin{equation*}
        f_n(x)=\sum_{m=0}^{2^n\times n} \frac{m}{2^n}\times 1_{\{\frac{m}{2^n}\leq f^n(x)<\frac{m+1}{2^n}\}},\;\;\; n\in\mathbb{N}^+.
    \end{equation*}
    由Ex 1.3.4的解 \ref{1.3.4},$\{f_n\}$是递增且有界的非负简单函数序列,同时$f$是$f_n$的上确界。记$\mathcal{D}_n=\{(x,t)\in \mathbb{R}^d\times \mathbb{R} : 0\leq t\leq f_n(x)\}$,显然$\mathcal{D}_n$是$\Rnum^{d+1}$上的递增可测集序列:
    \begin{equation*}
        \mathcal{D}_n\subset \mathcal{D}_{n+1},\;\;\; \forall n\in\mathbb{N}^+.
    \end{equation*}
    根据单调收敛定理,$\lim_{n\to\infty}\mathcal{D}_{n}$是$\Rnum^{d+1}$上的可测集,记为$\mathcal{D}_{\infty}$。由于$f$是序列$\{f_n\}$的上界,故
    \begin{equation*}
        \mathcal{D}_{\infty}\subset \mathcal{D}.
    \end{equation*}
    另一方面,由于$f$是序列$\{f_n\}$的上确界,故任给$(x,t)\in\mathcal{D}$(其中$t<f(x)$),存在$N\in\mathbb{N}$使得任给$n\geq N$,有$(x,t)\in \mathcal{D}_n\subset\mathcal{D}_{\infty}$。故
    \begin{equation*}
        (\mathcal{D}\setminus\{(x,f(x)):x\in\Rnum^d\})\subset \mathcal{D}_\infty.
    \end{equation*}
    又$m(\{(x,f(x)):x\in\Rnum^d\})=0$。故$m^*(\mathcal{D}\Delta \mathcal{D}_\infty)=0$,$\mathcal{D}$是可测集。
\end{proof}


\section{Solution of Ex 1.3.10}
(Basic properties of the lower Lebesgue integral). Let $f,g:\Rnum^d\to [0,+\infty]$ be unsigned functions (not necessarily measurable)

(i) (Compatibility with the simple integral) If $f$ is simple, then we have $\underline{\int_{\Rnum^d}} f(x)\dif x =  \overline{\int_{\Rnum^d}} f(x)\dif x= \mathrm{Simp}\int_{\Rnum^d} f(x)\dif x$.
\begin{proof}
    根据下积分的定义,令$h=f$,$h$显然是满足$h(x)\leq f(x)$条件的简单函数。故
    \begin{equation*}
        \underline{\int_{\Rnum^d}} f(x)\dif x\geq \mathrm{Simp}\int_{\Rnum^d} h(x)\dif x.
    \end{equation*}
    另一方面,任给简单函数$g(x)\leq f(x)$,显然有$g(x)\leq h(x)$成立,故
    \begin{equation*}
        \underline{\int_{\Rnum^d}} f(x)\dif x\leq \mathrm{Simp}\int_{\Rnum^d} h(x)\dif x.
    \end{equation*}
    又由于$\mathrm{Simp}\int_{\Rnum^d} h(x)\dif x=\mathrm{Simp}\int_{\Rnum^d} f(x)\dif x$。则我们可以得到
    \begin{equation*}
        \underline{\int_{\Rnum^d}} f(x)\dif x=\mathrm{Simp}\int_{\Rnum^d} f(x)\dif x.
    \end{equation*}
    同理,可以证明上积分的情况。
\end{proof}
(ii) (Monotonicity) If $f\leq g$ pointwise almost everywhere, then we have $\underline{\int_{\Rnum^d}}f(x)\dif x\leq \underline{\int_{\Rnum^d}}g(x)\dif x$ and $\overline{\int_{\Rnum^d}}f(x)\dif x\leq \overline{\int_{\Rnum^d}}g(x)\dif x$.
\begin{proof}
    任给简单函数$h\leq f$,显然有$h\leq g$,故
    \begin{equation*}
        \underline{\int_{\Rnum^d}}f(x)\dif x\leq \underline{\int_{\Rnum^d}}g(x)\dif x.
    \end{equation*}
    另一方面,任给简单函数$s\geq g$,显然有$s\geq f$,故
    \begin{equation*}
        \overline{\int_{\Rnum^d}}f(x)\dif x\leq \overline{\int_{\Rnum^d}}g(x)\dif x.
    \end{equation*}
\end{proof}
(iii) (Homogeneity) If $c\in [0,+\infty)$, then $\underline{\int_{\Rnum^d}}cf(x)\dif x=c\underline{\int_{\Rnum^d}}f(x)\dif x$.
\begin{proof}
    当$c=0$时,上式显然成立,我们接下来考虑$c\neq 0$的情形。任给简单函数$h\leq f$,显然有$c\times h(x)\leq c\times f(x)$成立,且$c\times h(x)$也是简单函数。故由(i)和(ii),我们有
    \begin{equation*}
        \underline{\int_{\Rnum^d}}cf(x)\dif x\geq \mathrm{Simp}\int_{\Rnum^d} ch(x)\dif x=c\times\mathrm{Simp}\int_{\Rnum^d} h(x)\dif x,\;\;\; \forall h\leq f,h \mathrm{\ is\ Simple}.
    \end{equation*}
    也就是
    \begin{equation}
        \underline{\int_{\Rnum^d}}cf(x)\dif x\geq c\underline{\int_{\Rnum^d}}f(x)\dif x.
        \label{4.1}
    \end{equation}
    另一方面,任给简单函数$h\leq cf$,显然有简单函数$\frac{1}{c} h\leq f$。故,我们有
    \begin{equation*}
        \underline{\int_{\Rnum^d}}f(x)\dif x\geq \mathrm{Simp}\int_{\Rnum^d} \frac{1}{c}h(x)\dif x=\frac{1}{c}\times\mathrm{Simp}\int_{\Rnum^d} h(x)\dif x,\;\;\; \forall h\leq f,h \mathrm{\ is\ Simple}.
    \end{equation*}
    也就是
    \begin{equation}
        c\underline{\int_{\Rnum^d}}f(x)\dif x\geq \underline{\int_{\Rnum^d}}cf(x)\dif x.
        \label{4.2}
    \end{equation}
    结合 \eqref{4.1} 和 \eqref{4.2},我们就得到了题设的结果。
\end{proof}
(iv) (Equivalence) If $f,g$ agree almost everywhere, then $\underline{\int_{\Rnum^d}}f(x)\dif x=\underline{\int_{\Rnum^d}}g(x)\dif x$ and $\overline{\int_{\Rnum^d}}f(x)\dif x=\overline{\int_{\Rnum^d}}g(x)\dif x$.
\begin{proof}
    假设$f,g$在零测集$E$之外处处相等。函数$m,M$由下式定义:
    \begin{equation*}
        \begin{split}
            m(x)=M(x)=f(x),\;\;\;\forall x\in \Rnum^d \setminus E,\\
            m(x)=0,\;\;\;M(x)=+\infty,\;\;\;\forall x\in E.
        \end{split}
    \end{equation*}
    显然,我们有$m\leq f,g\leq M$。任给简单函数$h\leq M$,考虑下列简单函数$h^\prime$:
    \begin{equation*}
        \begin{split}
            h^\prime(x)&=h(x),\;\;\;\forall x\in\Rnum^d \setminus E,\\
            h^\prime(x)&=0,\;\;\; \forall x\in E.
        \end{split}
    \end{equation*}
    显然,我们有$h^\prime\leq m$且$\mathrm{Simp}\int_{\Rnum^d} h^\prime(x)\dif x=\mathrm{Simp}\int_{\Rnum^d} h(x)\dif x$。由此,我们有
    \begin{equation*}
        \underline{\int_{\Rnum^d}}m(x)\dif x\geq \underline{\int_{\Rnum^d}}M(x)\dif x.
    \end{equation*}
    由上式和(ii),我们可以得到$\underline{\int_{\Rnum^d}}f(x)\dif x=\underline{\int_{\Rnum^d}}g(x)\dif x$。同理,$\overline{\int_{\Rnum^d}}f(x)\dif x=\overline{\int_{\Rnum^d}}g(x)\dif x$也成立。
\end{proof}
(v) (Superadditivity) $\underline{\int_{\Rnum^d}}f(x)+g(x)\dif x\geq \underline{\int_{\Rnum^d}}f(x)\dif x+\underline{\int_{\Rnum^d}}g(x)\dif x$.

(vi) (Subadditivity of upper integral) $\overline{\int_{\Rnum^d}}f(x)+g(x)\dif x\leq \overline{\int_{\Rnum^d}}f(x)\dif x+\overline{\int_{\Rnum^d}}f(x)\dif x$.
\begin{proof}
    任给两个简单函数$h_1,h_2$,使得$h_1\leq f,h_2\leq g$。显然,$h_1+h_2\leq f+g$,故我们有\\
    $\underline{\int_{\Rnum^d}}f(x)+g(x)\dif x\geq \underline{\int_{\Rnum^d}}f(x)\dif x+\underline{\int_{\Rnum^d}}g(x)\dif x$成立。
    
    类似地,任给两个简单函数$h_1,h_2$,使得$h_1\geq f,h_2\geq g$。显然,我们有$h_1+h_2\geq f+g$。因此\\
    $\overline{\int_{\Rnum^d}}f(x)+g(x)\dif x\leq \overline{\int_{\Rnum^d}}f(x)\dif x+\overline{\int_{\Rnum^d}}f(x)\dif x$。
\end{proof}
(vii) (Divisibility) For any measurable set $E$, one has $\underline{\int_{\Rnum^d}}f(x)\dif x =\underline{\int_{\Rnum^d}}f(x)1_{E}(x)\dif x+\underline{\int_{\Rnum^d}}f(x)1_{\Rnum^d\setminus E}(x)\dif x$.
\begin{proof}
    显然,$f(x)=f(x)1_{E}(x)+f(x)1_{\Rnum^d\setminus E}(x)$,故由(v),我们有
    \begin{equation}
        \underline{\int_{\Rnum^d}}f(x)\dif x \geq \underline{\int_{\Rnum^d}}f(x)1_{E}(x)\dif x+\underline{\int_{\Rnum^d}}f(x)1_{\Rnum^d\setminus E}(x)\dif x.
        \label{4.3}
    \end{equation}
    另一方面,任给简单函数$h\leq f$,有$h(x)=h(x)1_{E}(x)+h(x)1_{\Rnum^d\setminus E}(x)$成立,且两者都是简单函数。此外,显然有下式成立:
    \begin{equation*}
        \mathrm{Simp}\int_{\Rnum^d} h(x)\dif x=\mathrm{Simp}\int_{\Rnum^d} h(x)1_{E}(x)\dif x+\mathrm{Simp}\int_{\Rnum^d} h(x)1_{\Rnum^d\setminus E}(x)\dif x.
    \end{equation*}
    同时,我们有
    \begin{equation*}
        h(x)1_{E}(x)\leq f(x)1_{E}(x),\;\;\; h(x)1_{\Rnum^d\setminus E}(x)\leq f(x)1_{\Rnum^d\setminus E}(x),\;\;\;\forall x\in\Rnum^d.
    \end{equation*}
    由此,可以得到
    \begin{equation}
        \underline{\int_{\Rnum^d}}f(x)\dif x \leq \underline{\int_{\Rnum^d}}f(x)1_{E}(x)\dif x+\underline{\int_{\Rnum^d}}f(x)1_{\Rnum^d\setminus E}(x)\dif x.
        \label{4.4}
    \end{equation}
    结合 \eqref{4.3} 和 \eqref{4.4},我们可以得到题设的结果。
\end{proof}
(viii) (Horizontal truncation) As $n\to\infty$, $\underline{\int_{\Rnum^d}}\min(f(x),n)\dif x$ converges to $\underline{\int_{\Rnum^d}}f(x)\dif x$.
\begin{proof}
    我们首先考虑$\underline{\int_{\Rnum^d}}f(x)\dif x=A< +\infty$的情形。给定$\epsilon>0$,存在非负有界简单函数$g$使得$g\leq f $且
    \begin{equation*}
        \underline{\int_{\Rnum^d}}f(x)\dif x-\epsilon\leq \mathrm{Simp}\int_{\Rnum^d}g(x)\dif x.
    \end{equation*}
    由于$g$有界,则存在$N\in\mathbb{N}$使得
    \begin{equation*}
        \underline{\int_{\Rnum^d}}\min(g(x),n)\dif x=\mathrm{Simp}\int_{\Rnum^d}g(x)\dif x,\;\;\;\forall n\geq N.
    \end{equation*}
    由于单调性,我们有
    \begin{equation*}
        \begin{split}
            \underline{\int_{\Rnum^d}}\min(f(x),n)\dif x&\leq \underline{\int_{\Rnum^d}}\min(f(x),n+1)\dif x\leq \underline{\int_{\Rnum^d}}f(x)\dif x,\;\;\;\forall n\in\mathbb{N},\\
            \underline{\int_{\Rnum^d}}\min(f(x),n)\dif x&\geq \underline{\int_{\Rnum^d}}\min(g(x),n)\dif x,\;\;\;\forall n\geq N.
        \end{split}
    \end{equation*}
    由$\epsilon$的任意性,我们就得到了题设的结果。

    当$\underline{\int_{\Rnum^d}}f(x)\dif x=+\infty$时,任给$M>0$,存在非负有界简单函数$g$使得$g\leq f$且
    \begin{equation*}
        \mathrm{Simp}\int_{\Rnum^d}g(x)\dif x\geq M.
    \end{equation*}
    由于$g$有界,则存在$N\in\mathbb{N}$使得
    \begin{equation*}
        \underline{\int_{\Rnum^d}}\min(g(x),n)\dif x=\mathrm{Simp}\int_{\Rnum^d}g(x)\dif x,\;\;\;\forall n\geq N.
    \end{equation*}
    由于单调性,我们有
    \begin{equation*}
        \begin{split}
            \underline{\int_{\Rnum^d}}\min(f(x),n)\dif x&\leq \underline{\int_{\Rnum^d}}\min(f(x),n+1)\dif x\leq \underline{\int_{\Rnum^d}}f(x)\dif x,\;\;\;\forall n\in\mathbb{N},\\
            \underline{\int_{\Rnum^d}}\min(f(x),n)\dif x&\geq \underline{\int_{\Rnum^d}}\min(g(x),n)\dif x,\;\;\;\forall n\geq N.
        \end{split}
    \end{equation*}
    由$M$的任意性,我们就得到了题设的结果。
\end{proof}
(ix) (Vertical truncation) As $n\to \infty$, $\underline{\int_{\Rnum^d}}f(x)1_{\abs{x}\leq n}\dif x$ converges to $\underline{\int_{\Rnum^d}}f(x)\dif x$. \textit{Hint:} From Exercise 1.2.11 one has $m(E\cap\{x:\abs{x}\leq n\})\to m(E)$ for any measurable set $E$.
\begin{proof}
    我们首先考虑$\underline{\int_{\Rnum^d}}f(x)\dif x=A< +\infty$的情形。给定$\epsilon>0$,存在非负且支撑有限的简单函数$g$使得$g\leq f $且
    \begin{equation*}
        \underline{\int_{\Rnum^d}}f(x)\dif x-\epsilon\leq \mathrm{Simp}\int_{\Rnum^d}g(x)\dif x.
    \end{equation*}
    由于$g$支撑有限,则存在$N\in\mathbb{N}$使得
    \begin{equation*}
        \underline{\int_{\Rnum^d}}g(x)1_{\abs{x}\leq n}\dif x=\mathrm{Simp}\int_{\Rnum^d}g(x)\dif x,\;\;\;\forall n\geq N.
    \end{equation*}
    由于单调性,我们有
    \begin{equation*}
        \begin{split}
            \underline{\int_{\Rnum^d}}f(x)1_{\abs{x}\leq n}\dif x&\leq \underline{\int_{\Rnum^d}}f(x)1_{\abs{x}\leq n+1}\dif x\leq \underline{\int_{\Rnum^d}}f(x)\dif x,\;\;\;\forall n\in\mathbb{N},\\
            \underline{\int_{\Rnum^d}}f(x)1_{\abs{x}\leq n}\dif x&\geq \underline{\int_{\Rnum^d}}g(x)1_{\abs{x}\leq n}\dif x,\;\;\;\forall n\geq N.
        \end{split}
    \end{equation*}
    由$\epsilon$的任意性,我们就得到了题设的结果。类似地,可以证明$\underline{\int_{\Rnum^d}}f(x)\dif x=+\infty$的情况。
\end{proof}
(x) (Reflection) If $f+g$ is a simple function that is bounded with finite measure support(i.e. it is absolutely integrable), then we have $\mathrm{Simp}\int_{\Rnum^d} f(x)+g(x)\dif x=\underline{\int_{\Rnum^d}}f(x)\dif x+\overline{\int_{\Rnum^d}}g(x)\dif x$.
\begin{proof}
    
\end{proof}
Do the horizontal and vertical truncation properties hold if the lower Lebesgue integral is replaced with the upper Lebesgue integral?
\begin{proof}
    考虑下列函数$f$:
    \begin{equation*}
        \begin{split}
            f(x)&=+\infty,\;\;\;\forall x\in[0,1],\\
            f(x)&=0,\;\;\;\forall x\notin[0,1].
        \end{split}
    \end{equation*}
    它的任意水平截断都是0函数,故任给$n$,有
    \begin{equation*}
        \overline{\int_{\Rnum^d}}\min(f(x),n)\dif x=\underline{\int_{\Rnum^d}}\min(f(x),n)\dif x=0,\;\;\;\forall n\in\mathbb{N}.
    \end{equation*}
    另一方面,显然有$\overline{\int_{\Rnum^d}}f(x)\dif x\neq 0$。故水平截断对于上积分不成立。
    
    考虑下列函数$f$:
    \begin{equation*}
        \begin{split}
            f(x)&=+\infty,\;\;\;\forall x\in\mathbb{N},\\
            f(x)&=0,\;\;\;\forall x\notin\mathbb{N}.
        \end{split}
    \end{equation*}
    类似地,可以证明垂直截断性质对其的上积分不成立。
\end{proof}
\section{Solution of Ex 1.3.13}
(Area interpretation of integral). If $f:\Rnum^d\to[0,+\infty]$ is a measurable, show that $\int_{\Rnum^d} f(x)\dif x$ is equal to the $d+1$-dimensional Lebesgue measure of the region $\mathcal{D}=\{(x,t)\in\Rnum^d\times \Rnum:0\leq t\leq f(x)\}$. (This can be used as an alternate, and more geometrically intuitive, definition of the unsigned Lebesgue integral; it is a more convenient formulation for establishing the basic convergence theorems, but not quite as convenient for establishing basic properties such as additivity.) (\textit{Hint:} Use Exercies 1.2.22.)
\begin{proof}
    
\end{proof}




\section{Solution of Ex 1.3.16}
(Linear change of variables). Let $f:\Rnum^d\to[0,+\infty]$ be measurable, and let $T:\Rnum^d\to\Rnum^d$ be a invertible linear transformation. Show that $\int_{\Rnum^d} f(T^{-1}(x))\dif x = \abs{\mathrm{det}T}\int_{\Rnum^d} f(x)\dif x$, or equivalenctly, that $\int_{\Rnum^d} f(Tx)\dif x=\frac{1}{\mathrm{det}T}\int_{\Rnum^d} f(x)\dif x$.
\begin{proof}
    
\end{proof}

\section{Solution of Ex 1.3.17}
(Compatibility with the Riemann integral). Let $f:[a,b]\to [0,+\infty]$ be  Riemann integrable. If we extend $f$ to $\Rnum$ be declaring $f$ to equal zero outside of $[a,b]$, show that $\int_\Rnum f(x)\dif x=\int_a^b f(x)\dif x$.
\begin{proof}

\end{proof}




\section{Solution of Ex 1.3.24}
Show that a function $f:\Rnum^d\to \mathbb{C}$ is measurable if and only if it is pointwise almost everywhere limit of continuous functions $f_n:\Rnum^d\to \mathbb{C}$. (\textit{Hint:} If $f:\Rnum^d\to\mathbb{C}$ is measurable and $n\geq 1$, show that there exists a continuous function $f_n:\Rnum^d\to \mathbb{C}$ for which the set $\{x\in B(0,0):\abs{f(x)-f_n(x)}\geq 1/n\}$ has measure at most $\frac{1}{2^n}$. You may find Exercise 1.3.24 below to be useful for this.) Use this (and Egorov's theorem, Theorem 1.3.26) to give an alternate proof of Lusin's theorem for arbitrary measurable functions.
\begin{proof}

\end{proof}


\end{document}
