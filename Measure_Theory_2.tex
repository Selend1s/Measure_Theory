\documentclass[reqno,a4paper,14pt]{amsart}
\usepackage{amsmath}
\usepackage{amsthm}
\usepackage{ctex}
\usepackage{mathtools}
\newcommand\me{\mathrm{e}}
\newcommand\mi{\mathrm{i}}
\usepackage{mathtools}
%\usepackage[greek,english]{babel}
\usepackage{amssymb}
\newcommand\dif{\,\mathrm{d}}
\newcommand\mE{\mathbb{E}}
\newcommand\Pro{\mathbb{P}}
\newcommand{\Kov}{柯尔莫戈洛夫}
\makeatletter
\@addtoreset{equation}{section}
\makeatother
\renewcommand{\theequation}{\arabic{section}.\arabic{equation}}
\newcommand{\abs}[1]{\left\vert#1\right\vert}
\usepackage{hyperref}
\hypersetup{hypertex=true,
            colorlinks=true,
            linkcolor=blue,
            anchorcolor=black,
            citecolor=black}

\usepackage{mathrsfs}
%\usepackage{ntheorem}
%\theoremsymbol{\mbox{$\box$}}
\newtheorem{theorem}{定理}[section]
\newtheorem{lemma}[theorem]{引理}
\newtheorem{proposition}[theorem]{命题}
\newtheorem{corollary}[theorem]{推论}
\newtheorem{definition}[theorem]{定义}
\newtheorem{example}{例}
\newtheorem{remark}[theorem]{注}
\newtheorem*{prove}{\textbf{解}}
\newenvironment{solution}{\begin{proof}[\indent\bf 解]}{\end{proof}}
\renewcommand{\proofname}{\indent\bf 证明}
\newtheorem*{thm2}{Proof of Theorem 2.2}
\usepackage{geometry}
\geometry{a4paper,left=2.5cm,right=2.5cm,top=3cm,bottom=3cm}


\title{\textbf{测度论导论第一章第一节习题}}
\author{丁\;\;\; 珍}
    % \address{201940100486}
\author{程预敏}
    % \address{201940100447}

\begin{document}
\maketitle
\section{Solution of EX 1.2.11}
(i) (Upward monotone convergence) Let $E_1\subset E_2\subset \dots \subset \mathbb{R}^n$ be a countable non-decreasing sequence of Lebesgue measurable sets. Show shta $m(\cup_{n=1}^\infty E_n)=\lim_{n\to\infty} m(E_n)$. (\textit{Hint:} Express $\cup_{n=1}^\infty E_n$ as the countable union of the lacunae $E_n\backslash \cup_{n^\prime=1}^{n-1} E_{n^\prime}$.)
\begin{proof}
    
\end{proof}
(ii) (Downward monotone convergence) Let $\mathbb{R}^d \supset E_1 \supset E_2\supset \dots$ be a countable non-increasing sequence of Lebesgue measurable sets. If at least one of the $m(E_n)$ is finite, show that $m(\cap_{n=1}^\infty E_n)=\lim_{n\to \infty} m(E_n)$.
\begin{proof}
    
\end{proof}
(iii) Give a counterexample to show that in the hypothesis that at least one of the $m(E_n)$ is finite in the downward monotone convergence theorem cannot be dropped.
\begin{proof}
    
\end{proof}


\section{Solution of EX 1.2.12}
Show that any map $E\to m(E)$ from Lebesgue measurable sets to elements of $[0,+\infty]$ that obeys the above empty set and countable additivity axioms will also obey the monotinicity and countable subadditivity axioms from Exercise 1.2.3, when restricted to Lebesgue measurable sets of course.
\begin{proof}
    
\end{proof}


\section{Solution of EX 1.2.22}
Let $d,d^\prime\geq 1$ be natural numbers. (i) If $E\subset \mathbb{R}^{d^\prime}$, show that 
\begin{equation*}
    (m^{d+d^\prime})^*(E\times F)\leq (m^d)^*(E)\times (m^{d^\prime})^*(F),
\end{equation*}
where $(m^d)^*$ denotes $d$-dimensional Lebesgue measure, etc.
\begin{proof}
    
\end{proof}
(ii) Let $E\subset \mathbb{R}^d, F\subset \mathbb{R}^{d^\prime}$ be Lebesgue measurable sets. Show that $E\times F\subset \mathbb{R}^{d+d^\prime}$ is Lebesgue measurable, with $m^{d+d^\prime}(E\times F)=m^d(E) \dot m^{d^\prime}(F)$. (Note that we allow $E$ or $F$ to have infinite measure, and so one may have to divide into cases or take advantage of the monotone convergence theorem for Lebesgue measure, Exercise 1.2.11)
\begin{proof}
    
\end{proof}


\section{Solution of EX 1.2.23}
(Uniqueness of Lebesgue measure). Show that Lebesgue measure $E\to m(E)$ is the only map from Lebesgue measurable sets to $[0,+\infty]$ that obeys the following axioms:

(i) (Empty set) $m(\emptyset) =0$.

(ii) (Countable additivity) If $E_1,E_2,\dots \subset \mathbb{R}^d$ is a countable sequence of disjoint Lebesgue measurable sets, then $m(\cup_{n=1}^\infty E_n)\sum_{n=1}^\infty m(E_n)$.

(iii) (Translation invariance) If $E$ is Lebesgue measurable and $x\in \mathbb{R}^d$, then $m(E+x)=m(E)$.

(iv) (Normalisation) $m([0,1]^d)=1$.
\begin{proof}
    
\end{proof}



\end{document}
