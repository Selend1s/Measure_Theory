\documentclass[reqno,a4paper,14pt]{amsart}
\usepackage{amsmath}
\usepackage{amsthm}
\usepackage{ctex}
\usepackage{mathtools}
\newcommand\me{\mathrm{e}}
\newcommand\mi{\mathrm{i}}
\usepackage{mathtools}
%\usepackage[greek,english]{babel}
\usepackage{amssymb}
\newcommand\dif{\,\mathrm{d}}
\newcommand\mE{\mathbb{E}}
\newcommand\Pro{\mathbb{P}}
\newcommand{\Kov}{柯尔莫戈洛夫}
\makeatletter
\@addtoreset{equation}{section}
\makeatother
\renewcommand{\theequation}{\arabic{section}.\arabic{equation}}
\newcommand{\abs}[1]{\left\vert#1\right\vert}
\usepackage{hyperref}
\hypersetup{hypertex=true,
            colorlinks=true,
            linkcolor=blue,
            anchorcolor=black,
            citecolor=black,
            pdfauthor={丁珍,程预敏},%
            pdftitle={测度论练习题2},%
            pdfsubject={概率论},
            pdfkeywords={测度论,Lebesgue测度},
            pdfproducer={LaTeX},%
            pdfcreator={xeLaTeX}}

\usepackage{mathrsfs}
%\usepackage{ntheorem}
%\theoremsymbol{\mbox{$\box$}}
\newtheorem{theorem}{定理}[section]
\newtheorem{lemma}[theorem]{引理}
\newtheorem{proposition}[theorem]{命题}
\newtheorem{corollary}[theorem]{推论}
\newtheorem{definition}[theorem]{定义}
\newtheorem{example}{例}
\newtheorem{remark}[theorem]{注}
\newtheorem*{prove}{\textbf{解}}
\newenvironment{solution}{\begin{proof}[\indent\bf 解]}{\end{proof}}
\renewcommand{\proofname}{\indent\bf 证明}
\newtheorem*{thm2}{Proof of Theorem 2.2}
\usepackage{geometry}
\geometry{a4paper,left=2.5cm,right=2.5cm,top=3cm,bottom=3cm}


\title{\textbf{测度论导论\S1.3习题}}
\author{丁\;\;\; 珍}
    % \address{201940100486}
\author{程预敏}
    % \address{201940100447}

\begin{document}
\maketitle
\section{Solution of Ex 1.2.11}
(i) (Upward monotone convergence) Let $E_1\subset E_2\subset \dots \subset \mathbb{R}^n$ be a countable non-decreasing sequence of Lebesgue measurable sets. Show shta $m(\bigcup_{n=1}^\infty E_n)=\lim_{n\to\infty} m(E_n)$. (\textit{Hint:} Express $\bigcup_{n=1}^\infty E_n$ as the countable union of the lacunae $E_n\backslash \bigcup_{n^\prime=1}^{n-1} E_{n^\prime}$.)
\begin{proof}
    记$E=\bigcup_{n=1}^\infty E_n$。对于任意$k\geq 2$,令
    \begin{equation*}
        G_1=E_1,\;\;\; G_2=E_2-E_1, \cdots, G_k=E_k-E_{k-1}.
    \end{equation*}
    根据Lemma 1.2.13,对于任意$k\geq 1$,$G_k$ 是两两不交的可测集,且
    \begin{equation*}
        E_n=\bigcup_{k=1}^n G_k,\;\;\; E=\bigcup_{k=1}^\infty G_k.
    \end{equation*}
    因此
    \begin{equation*}
        m(\bigcup_{n=1}^\infty E_n)=m(E)=\sum_{k=1}^\infty m(G_k)=\lim_{N\to\infty} \sum_{k=1}^N m(G_k)=\lim_{N\to\infty}m\biggl(\bigcup_{k=1}^N G_k\biggr)=\lim_{N\to\infty}m(E_N).
    \end{equation*}
\end{proof}
(ii) (Downward monotone convergence) Let $\mathbb{R}^d \supset E_1 \supset E_2\supset \dots$ be a countable non-increasing sequence of Lebesgue measurable sets. If at least one of the $m(E_n)$ is finite, show that $m(\bigcap_{n=1}^\infty E_n)=\lim_{n\to \infty} m(E_n)$.
\begin{proof}
    记$E=\bigcap_{n=1}^\infty E_n$。不失一般性,我们假设$m(E_1)<\infty$。令
    \begin{equation*}
        G_1=E_1-E_2,\;\;\; G_2=E_2-E_3, \cdots, G_k=E_k-E_{k+1}.
    \end{equation*}
    根据Lemma 1.2.13,对于任意$k\geq 1$,$G_k$ 是两两不交的可测集,且
    \begin{equation*}
        E_1=E\cup \bigcup_{k=1}^\infty G_k
    \end{equation*}
    是一个可测集的不交并。由此,我们有
    \begin{equation*}
        \begin{split}
            m(E_1)&=m(E)+\sum_{k=1}^\infty m(G_k)=m(E)+\lim_{N\to\infty} \sum_{k=1}^N m(G_k)\\
            &=m(E)+\lim_{N\to\infty}\sum_{k=1}^N\bigl(m(E_k)-m(E_{K+1})\bigr)\\
            &=m(E)+m(E_1)-\lim_{N\to\infty}m(E_{N+1}).
        \end{split}
    \end{equation*}
    由于$m(E_1)<\infty$,且对于任意$k>1$,有$E_k\subset E_1,m(E_k)<\infty$。综上,我们有
    \begin{equation*}
        m(E)=m(\bigcap_{n=1}^\infty E_n)=\lim_{N\to\infty}m(E_{N+1})=\lim_{N\to\infty}m(E_{N}).
    \end{equation*}
\end{proof}
(iii) Give a counterexample to show that in the hypothesis that at least one of the $m(E_n)$ is finite in the downward monotone convergence theorem cannot be dropped.
\begin{proof}
    令$E_n=(n,\infty)\subset \mathbb{R}$。对于任意$k\geq 1$,我们有$m(E_K)=\infty$。同时,令$E=\bigcap_{n=1}^\infty E_n$,$\forall x\in \mathbb{R}$,存在$N\in\mathbb{N}$,使得
    \begin{equation*}
        x<N,\;\;\; x\notin E_N.
    \end{equation*}
    故$E=\emptyset,m(E)=0$。综上$m(E)=m(\bigcap_{n=1}^\infty E_n)\neq \lim_{n\to\infty}m(E_n)$。
\end{proof}




\section{Solution of Ex 1.2.12}
Show that any map $E\to m(E)$ from Lebesgue measurable sets to elements of $[0,+\infty]$ that obeys the above empty set and countable additivity axioms will also obey the monotinicity and countable subadditivity axioms from Exercise 1.2.3, when restricted to Lebesgue measurable sets of course.
\begin{proof}
    (i) 单调性。 令$E,G$是两个Lebesgue可测集,且$G\subset E\subset \mathbb{R}^d$。设$f$是从Lebesgue可测集到$\overline{\mathbb{R}^+}$的满足空集和可数可加性公理的映射。不妨设$f(G)<\infty$。根据可数可加性和空集公理,我们有
    \begin{equation*}
        f(E)=f(G)+f(E\backslash G).
    \end{equation*}
    由于$E\backslash G$是Lebesgue可测集,则$f(E\backslash G)\geq 0$。于是,我们有$f(E)\geq f(G)$。若$f(G)=\infty$,显然有$f(E)=\infty\geq f(G)$。

    (ii) 可数次可加性。令$E_1,E_2,\cdots \subset \mathbb{R}^d$是一个可数的Lebesgue可测集序列,且对于任意$n\in\mathbb{N}^+$有$f(E_n)<\infty$。同时记$\bigcup_{n=1}^\infty E_n=E$。此外,令Lebesgue可测集序列$\{G_k\}$由下定义:
    \begin{equation*}
        G_1=E_1,\;\;\; G_k=E_k\backslash\bigcup_{n^\prime=1}^{k-1} E_{n^\prime}:\;\; \forall k\geq 2.
    \end{equation*}
    显然,$\{G_k\}$是可数的不交的Lebesgue可测集序列,且对于任意$N\in \mathbb{N}^+$,有$\bigcup_{k=1}^N E_k=\bigcup_{k=1}^N G_k$。由$f$的可数可加性,我们有
    \begin{equation*}
        f(E)=f(\bigcup_{k=1}^\infty E_k)=f(\bigcup_{k=1}^\infty G_k)=\lim_{N\to\infty}\sum_{k=1}^N f(G_k).
    \end{equation*}
    另一方面,对于任意$N\in \mathbb{N}^+$,有$G_k\subset E_k$,根据$f$的单调性,我们有
    \begin{equation*}
        f(E_k)\geq f(G_k):\;\;\;\forall k\in \mathbb{N}^+.
    \end{equation*}
    综上,我们有
    \begin{equation*}
        f(E)=f(\bigcup_{k=1}^\infty E_k)=\lim_{N\to\infty}\sum_{k=1}^N f(G_k)\leq \lim_{N\to\infty}\sum_{k=1}^N f(E_k).
    \end{equation*}
    若$\exists n_0\in \mathbb{N}^+$使得$f(E_{n_0})=\infty$,显然有$f(E)\leq \sum_{k=1}^\infty f(E_k)$成立。
\end{proof}

\section{Solution of Ex 1.2.14}
Let $E\subset \mathbb{R}^d$. Show that $E$ is contained in a Lebesgue measurable set of measure exactly equal to $m^*(E)$.
\begin{proof}
    若$m^*(E)=\infty$,令$U=\mathbb{R}^d\supset E$,$m(U)=\infty=m^*(E)$成立。
    
    若$m^*(E)<\infty$。由外测度定义,任给$\epsilon>0$和$i\in\mathbb{N}^+$,存在开集$U_i\supset E$,且$m(U_i)\leq m^*(E)+\frac{1}{i}$。
    令$S=\bigcap_{i=1}^\infty U_i$是可测集。显然,对于任意$i\in\mathbb{N}^+$,$U_i\supset E$,故$S\supset E$,$m(S)=m^*(S)\geq m^*(E)$。另一方面,对于任意$\epsilon>0$,存在$i_\epsilon\in \mathbb{N}$,使得$\frac{1}{i_\epsilon}\leq \epsilon$。这时,我们有
    \begin{equation*}
        S\subset U_{i_\epsilon},\;\;\; m(S)\leq m(U_{i_\epsilon})\leq m^*(E)+\frac{1}{i_\epsilon}\leq m^*(E)+\epsilon.
    \end{equation*}
    由$\epsilon$的任意性,我们有$S\supset E$且$m(S)=m^*(E)$。
\end{proof}


\section{Solution of Ex 1.2.15}
(Inner regularity). Let $E\subset \mathbb{R}^d$ be Lebesgue measurable. Show that 
\begin{equation*}
    m(E)=\sup_{K\subset E,\ K\ \mathrm{compact}} m(K).
\end{equation*}
\begin{proof}
    对于$E$的任意紧子集$K$,$m(E)\geq m(K)$显然成立。故我们下面证明
    \begin{equation}
        m(E)\leq\sup_{K\subset E,\ K\ \mathrm{compact}} m(K)
        \label{bound}
    \end{equation}

    (1) 当$E$是有界集时。令$C$是包含$E$的一个有界闭集,对于任意$\epsilon>0$。由Lebesgue测度的外正则性,存在一个开集$U\supset (C\backslash E)$,且
    \begin{equation}
        m(U)<m(C\backslash E) +\epsilon.
        \label{open}
    \end{equation}
    令$K=C\backslash U$,则$K$是一个$E$的紧子集。此外,根据Lebesgue测度的单调性,我们有
    \begin{equation}
        m(C)\leq m(K)+m(U).
        \label{close}
    \end{equation}
    综合 \eqref{open} 和 \eqref{close}两式,我们可以得到
    \begin{equation*}
        \begin{split}
            m(E)&+m(C\backslash E)=m(C)\leq m(K)+m(C\backslash E) +\epsilon,\\
            m(E)&\leq m(K)+\epsilon.
        \end{split}
    \end{equation*}
    根据$\epsilon$的任意性,我们可以得到 \eqref{bound}。

    (2) 当$E$不是有界集时。对于任意$M<m(E)$,我们可以找到$E$的一个紧子集$K$,且$m(K)>M$:\\令$E_i=E\bigcap \overline{B_i(O)}$,其中$\overline{B_i(O)}$是以原点为中心,半径为$i$的闭球。则对于任意$i\in\mathbb{N}^+$,
    \begin{equation*}
        E_i\subset E_{i+1},\;\;\; m(E)=\lim_{i\to\infty} m(E_i).
    \end{equation*}
    故存在一个$i_0$使得$m(E_{i_0})>M$,且$E_{i_0}$有界。由(1)可以找到紧集$K\subset E_{i_0}\subset E$使得$m(K)>M$。根据$M$的任意性,我们可以得到 \eqref{bound}。

\end{proof}
\section{Solution of Ex 1.2.22}
Let $d,d^\prime\geq 1$ be natural numbers. (i) If $E\subset \mathbb{R}^d$ and $F\subset \mathbb{R}^{d^\prime}$, show that 
\begin{equation}
    (m^{d+d^\prime})^*(E\times F)\leq (m^d)^*(E)\times (m^{d^\prime})^*(F),
    \label{outer_ineq}
\end{equation}
where $(m^d)^*$ denotes $d$-dimensional Lebesgue outer measure, etc.      
\begin{proof}
    (1) 若$E,F$中有一集合外测度为$\infty$且另一集合外测度不为零,则上式显然成立。
    
    (2) 故我们首先考虑$E,F$的Lebesgue外测度均有限的情况。令$O_E,O_F$是两个开集,且对于任意$\epsilon>0$满足如下条件
    \begin{equation*}
        \begin{split}
            O_E&\supset E,\;\;\; (m^d)(O_E)=(m^d)^*(O_E)\leq (m^d)^*(E)+\epsilon;\\
            O_F&\supset F,\;\;\; (m^{d^\prime})(O_F)=(m^{d^\prime})^*(O_F)\leq (m^{d^\prime})^*(F)+\epsilon.
        \end{split}
    \end{equation*}
    根据Lemma 1.2.11,开集$O_E,O_F$可以表示为几乎不交的闭的cubes的可数并
    \begin{equation*}
        O_E=\bigcup_{i=1}^\infty B_i,\;\;\; O_F=\bigcup_{j=1}^\infty B^\prime_j,
    \end{equation*}
    其中$B_i$是$\mathbb{R}^d$中的闭cube,$B^\prime_j$是$\mathbb{R}^{d^\prime}$中的闭cube。任给$x\in E\times F\subset \mathbb{R}^{d+d^\prime}$,$x\in O_E\times O_F$。则$E\times F$是开集$O_E\times O_F$的子集,故
    \begin{equation*}
        (m^{d+d^\prime})^*(E\times F)\leq (m^{d+d^\prime})^*(O_E\times O_F)=(m^{d+d^\prime})^*(\bigcup_{i,j=1}^\infty B_i\times B^\prime_j).
    \end{equation*}
    由于$B_i,B^{\prime}_j$都是几乎不交的闭的cube,则
    \begin{equation*}
        \begin{split}
            (m^{d+d^\prime})^*(\bigcup_{i,j=1}^\infty B_i\times B^\prime_j)&=\sum_{i=1}^\infty \sum_{j=1}^\infty (m^d)^*(B_i)\cdot (m^{d^\prime})^*(B^\prime_j)=(m^d)^*(O_E)\cdot (m^{d^\prime})^*(O_F)\\
            &\leq (m^d)^*(E)\times (m^{d^\prime})^*(F)+\epsilon\cdot\bigl((m^d)^*(E)+(m^{d^\prime})(F)\bigr)+\epsilon^2.
        \end{split}
    \end{equation*}
    由$\epsilon$的任意性,我们有 \eqref{outer_ineq} 成立。
    
    (3) 若$E,F$中有一集合外测度为$\infty$且另一集合外测度为零,不妨设$(m^d)^*(E)=\infty$且$(m^{d^\prime})^*(F)=(m^{d^\prime})(F)=0$,这时,记$\{K_n\}$为如下序列
    \begin{equation*}
        K_n=F\cap B_n(O),\;\;\;\forall n\in\mathbb{N}^+,
    \end{equation*}
    其中$B_n(O)$是以原点为中心,半径为$n$的开球。这时,我们有
    \begin{equation*}
        F=\bigcup_{n=1}^\infty K_n,\;\;\; K_n\subset K_{n+1},\;\;\; \forall n\geq 1.
    \end{equation*}
    令$O_E\supset E$是一个开集,则
    \begin{equation}
        \begin{split}
            (m^{d+d^\prime})^*(E\times F)&\leq(m^{d+d^\prime})(O_E\times F)\\
            &=(m^d)^*(O_E)\cdot (m^{d^\prime})^*(F)\\
            &=(m^d)^*(O_E)\cdot \lim_{n\to\infty}(m^{d^\prime})^*(K_n)\\
            &=(m^d)^*(O_E)\cdot 0=0.
            \label{zero}
        \end{split}
    \end{equation}
\end{proof}

(ii) Let $E\subset \mathbb{R}^d, F\subset \mathbb{R}^{d^\prime}$ be Lebesgue measurable sets. Show that $E\times F\subset \mathbb{R}^{d+d^\prime}$ is Lebesgue measurable, with $m^{d+d^\prime}(E\times F)=m^d(E) · m^{d^\prime}(F)$. (Note that we allow $E$ or $F$ to have infinite measure, and so one may have to divide into cases or take advantage of the monotone convergence theorem for Lebesgue measure, Exercise 1.2.11.)
\begin{proof}
    (1) 我们从一个有限测度的特殊情况开始。假设$E,F$是两个可测集,且$m(E),m(F)<\infty$。对于任意$\epsilon>0$,根据上一小题的结论,存在开集$O_E,O_F$使得
    \begin{equation*}
        \begin{split}
            O_E&\supset E,\;\;\; (m^d)(O_E)\leq (m^d)(E)+\epsilon;\\
            O_F&\supset F,\;\;\; (m^{d^\prime})(O_F)\leq (m^{d^\prime})(F)+\epsilon.
        \end{split}
    \end{equation*}
    另一方面,存在紧集$K_E,K_F$使得
    \begin{equation*}
        \begin{split}
            K_E&\subset E,\;\;\; (m^d)(K_E)\geq (m^d)(E)-\epsilon;\\
            K_F&\subset F,\;\;\; (m^{d^\prime})(K_F)\geq (m^{d^\prime})(F)-\epsilon.
        \end{split}
    \end{equation*}
    显然,我们有$(K_E\times K_F)\subset (E\times F)\subset (O_E\times O_F)$,且$K_E\times K_F$是$\mathbb{R}^{d+d^\prime}$中的紧集,故它可测,同时
    \begin{equation*}
        (m^{d+d^\prime})(K_E\times K_F)\geq (m^d)(E)\times (m^{d^\prime})(F)-\epsilon\cdot\bigl((m^d)(E)+(m^{d^\prime})(F)\bigr)+\epsilon^2.
    \end{equation*}
    另一方面,$(O_E\times O_F)\backslash (E\times F)\subset (O_E\times O_F)\backslash (K_E\times K_F)$,且
    \begin{equation*}
        \begin{split}
            (m^{d+d^\prime})^*\bigl((O_E\times O_F)\backslash (E\times F)\bigr)&\leq (m^{d+d^\prime})\bigl((O_E\times O_F)\backslash (K_E\times K_F)\bigr)\\
            &=(m^{d+d^\prime})(O_E\times O_F)-(m^{d+d^\prime})(K_E\times K_F)\\
            &\leq 2\epsilon\cdot\bigl((m^d)(E)+(m^{d^\prime})(F)\bigr).
        \end{split}
    \end{equation*}
    由$\epsilon$的任意性,我们可知$E\times F$是$\mathbb{R}^{d+d^\prime}$中的可测集,且
    \begin{equation*}
        (m^{d+d^\prime})(E\times F)=(m^d)(E)\times (m^{d^\prime})(F).
    \end{equation*}
    (2) 由 \eqref{zero} 可知,若$E,F$中有一集合测度为$\infty$且另一集合测度为零,则
    \begin{equation*}
        (m^{d+d^\prime})(E\times F)=(m^{d+d^\prime})^*(E\times F)=(m^d)(E)\times (m^{d^\prime})(F)=0.
    \end{equation*}
    (3) 若$E,F$中有一集合测度为$\infty$且另一集合测度不为零,不妨设$(m^d)^*(E)=\infty$且$(m^{d^\prime})(F)\neq 0$,这时令序列$\{S_n\}$如下定义:
    \begin{equation*}
        S_n=E\cap B_n(O),\;\;\; n=1,2,\cdots
    \end{equation*}
    这时,我们有$S_n\subset S_{n+1},\forall n\geq 1$, 且$E=\bigcup_{n=1}^\infty S_n$。同时,令$K_F\subset F$是一紧集,且$(m^{d^\prime})(K_F)>0$。这时,我们有
    \begin{equation*}
        \begin{split}
            (m^{d+d^\prime})(E\times K_F)=\lim_{n\to\infty}(m^{d+d^\prime})(S_n\times K_F)=\lim_{n\to\infty}(m^d)(S_n)\times (m^{d^\prime})(K_F)=\infty.
        \end{split}
    \end{equation*}
    由单调性,我们可以得到
    \begin{equation*}
        (m^{d+d^\prime})(E\times F)=(m^d)(E)\times (m^{d^\prime})(F)=\infty.
    \end{equation*}
\end{proof}


\section{Solution of Ex 1.2.23}
(Uniqueness of Lebesgue measure). Show that Lebesgue measure $E\to m(E)$ is the only map from Lebesgue measurable sets to $[0,+\infty]$ that obeys the following axioms:

(i) (Empty set) $m(\emptyset) =0$.

(ii) (Countable additivity) If $E_1,E_2,\dots \subset \mathbb{R}^d$ is a countable sequence of disjoint Lebesgue measurable sets, then $m(\cup_{n=1}^\infty E_n)=\sum_{n=1}^\infty m(E_n)$.

(iii) (Translation invariance) If $E$ is Lebesgue measurable and $x\in \mathbb{R}^d$, then $m(E+x)=m(E)$.

(iv) (Normalisation) $m([0,1]^d)=1$.
\begin{proof}
    设$f:E\to f(E)$是一个从Lebesgue可测集到$[0,+\infty]$的映射,且满足上面的测度公理。由Exercise 1.2.8可知,基本集是Lebesgue可测集,且对于任意基本集$E$,$m(E)=m^{(J)}(E)$。另一方面,由Exercise 1.1.3可知,基本集上满足非负性、有限可加性和平移不变性的测度具有唯一性,且在相差一个常数系数的情况下等价。因此,我们考虑将$f$限制在基本集上。$f([0,1]^d)=1$,且$f$满足上面的三条性质,故$f$在基本集上与Lebesgue测度$m$等价。

    设$E$是一个开集,则$E=\bigcup_{i=1}^\infty Q_i$,其中$Q_i$是$d$维几乎不交的闭cube。由于$Q_i$是几乎不交的闭cube,则$Q_i$是基本集,且
    \begin{equation*}
        f(Q_i)=m(Q_i),\;\;\; f(E)=\sum_{i=1}^\infty f(Q_i)=\sum_{i=1}^\infty m(Q_i)=m(E).
    \end{equation*}
    故$f$和$m$在开集上等价。

    设$E$是一个Lebesgue可测集,由Lebesgue外测度的定义,我们有
    \begin{equation*}
        m(E)=m^*(E)=\inf_{E\subset U,\ U\ \mathrm{open}} m(U).
    \end{equation*}
    对任意$\epsilon>0$,令$S\supset E$是一个开集,且$m(S)\leq m(E)+\epsilon$。故
    \begin{equation}
        f(E)\leq f(S)=m(S)\leq m(E)+\epsilon.
        \label{upper}
    \end{equation}
    由$\epsilon$的任意性,我们有$f(E)\leq m(E)$。另一方面
    \begin{equation}
        f(E)=f(S)-f(S\backslash E)=m(S)-f(S\backslash E)\geq m(S)-m(S\backslash E)=m(E).
        \label{lower}
    \end{equation}
    结合 \eqref{upper} 和 \eqref{lower},我们可知,$f$和勒贝格测度$m$在可测集上等价。
\end{proof}



\end{document}
